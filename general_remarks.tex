\section{Further implications}

Another thing to note, is that by combining substitution of \autoref{thm:eberhard} with \autoref{thm:eberhard:extended} some of the previously shown theorems generalize to arbitrary closed orientable 2-manifolds. For these one has to deduce an equation similarly to  \autoref{eq:valence:3} from Eulers relation for cell complex decompositions on a closed connected orientable $2$-manifold. Let $v$, $e$ and $f$ be the number of vertices ($0$-cells), edges ($1$-cells) and faces ($2$-cells) of the decomposition. Then the Euler relation gives $v + f - e = 2(1-g)$ for a manifold of genus $g$. As seen previously in a similar way this can be transformed to
\begin{align}
  \sum_{k=3}^n \left(6 - k \right) p_k = 12(1-g) \label{eq:valence:3:manifold}
\end{align}
for valence $3$ and 
\begin{align}
  \sum_{k=3}^n \left(4 - k \right) p_k = 8(1-g)  \label{eq:valence:4:manifold}
\end{align}
for valence $4$. With these considerations one can now state:
\begin{theorem}
  Let $g \in \nats$ and $p = (p_3, p_4, p_5, \dots, p_n)$ be a given sequence satisfying \autoref{eq:valence:3:manifold}. Then the following holds:
  \begin{itemize}
  \item There exists $r \in \nats$ for which $p + r [5, 7]$ $3$-realizes the orientable closed compact 2-manifold of genus $g$.
  \item There exists $r \in \nats$ for which $p + r [2 \times 5, 8]$ $3$-realizes the orientable closed compact 2-manifold of genus $g$.
  \end{itemize}
  \begin{proof}
    Let $P$ be the realization of $p$ via \autoref{thm:eberhard:extended}. In the first case use \autoref{thm:construction:5:7} on $P$, in the second case use \autoref{thm:construction:5:8}
  \end{proof}
\end{theorem}

One could try to further expand these results to the other cases in the thesis. For this one would have either to proof a theorem similar to \autoref{thm:eberhard:extended} for valence $4$ and $5$ or one could try to explicitly build in handles for the $2$-manifold for a direct realization. The letter approach is very likely to proceed in the $4$-valent case of adding triangles and pentagons. Another interesting topic is proving classes of theorems of general Eberhards type. One could hope for a mechanism resembling the one used here extensively: Use an established Eberhard-like theorem for a first realization, replace each face by a self-fitting expansion of it and replace every unwanted face in some way. This mechanism should then allow for some kind of alteration to many different sequences of polygons for the replacements. A last interesting question is the condition of $q$-$r$-realization in the cases where there are some non-realizable sequences. In \autoref{thm:case3:6:main} there are some necessary conditions given. It would be interesting to lower the bounds even further or to strengthen the result in case these are the lower bounds.