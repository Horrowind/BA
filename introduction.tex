The thesis considers problems regarding the existence of polyhedra where for each $k \in \nats$ the number of $k$-gons in the polyhedron is given. The sequence of these numbers will be called the $p$-vector of the polyhedron. A basic question one could for example ask is ``Does there exist a polyhedron with exactly twelve pentagons and a hexagon?'' After several tries of building a solid with these conditions, one may come to the conclusion that no such polyhedron exists, and rightly so! From this assertion, further investigation could be done to search for a deeper reason for the existence or non-existence of a polyhedron for a given $p$-vector. One could try to find an exact criterion on the $p$-vector for realizability, but this seems to be a hard and out of scope problem. Viewing the problem from another angle is possible by using a combinatorical theorem known as ``Euler's relation''. This relation at first give no information about distinctive face counts for different number of sides, but by fixing some regularity of the realization and using several easy geometric properties of polygons one is able to transform ``Euler's relation'' to an interesting equation. The equation seems to imply that each face type has an inherent curvature which applies to the polyhedron. So for example triangles have a positive curvature and ``close'' the polyhedron, while heptagons and larger $k$-gons have negative curvature and seem to ``widen'' the realization. This equation only gives a necessary condition for realizability of the polyhedron, so there are nonetheless $p$-vectors satisfying this equation, but for which no realization exists. In view of this, the equation admits the possibility to add in some more polygons which act flat - they give in sum no positive or negative curvature. So while some sequences are not realizable, maybe it is possible to add some more polygons with no total curvature and try realize this sequence. A theorem which states that this is always possible (where the exact meaning of ``this'' will be defined later) will be called a general Eberhard's theorem. This thesis proves some of these theorems positively and also gives some examples where adding certain polygons does not help in the realization. The structure of this thesis is as following: The first two sections specify the conditions and equations stated above and introduce notation and the arguments used. The next five sections are devoted to a general Eberhard's theorem and prove them fully where possible. The last two sections give conditions where one can not hope for a general Eberhard's theorem and insight to some extensions to the present work. Each proof uses one or more figures to describe a construction. The general reading on most of them is ``the image shown left is transformed to the one on the right'' if the image is about to describe a specific transformation.
