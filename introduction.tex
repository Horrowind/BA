The thesis discusses some problems regarding the existence of a polyhedron with a given amount of polygons of certain type. Here one is not as much interested in types of polygons, as in as regular or equilateral polygons, the only distinction to be made is how many sides a polygon has. So when talking about a polyhedron, the main interest here is in the number of triangles, the number of quadrangles and so on. The sequence of these numbers will be called the $p$-vector of the polyhedron. A basic question one could ask is ``Does there exist a polyhedron with exactly twelve pentagons and a hexagon?'' After several tries of building a solid with these conditions, one may come to the conclusion that such a polyhedron exists, and rightly so! One could try to find a deeper reason for this and try to find an exact criterion on how each number of each polygon type is responsible for a possible realization, but this seems to be a hard and out of scope problem. Viewing the problem from another angle is possible by using a combinatorical theorem known as ``Euler's relation''. These at first give no information about distinctive face counts for different number of sides, but by fixing some regularity of the realization and using several easy geometric properties of polygons one is able to transform ``Euler's relation'' to an interesting equation. The equation seem to imply that each face type seems to have an inherent curvature which it gives to the polyhedron. So for example triangles have a positive curvature and close the polyhedron in some way, while heptagons and larger $k$-gons have negative curvature and seem to widen the realization. This equation only gives a necessary condition for realizability of the polyhedron, so there are nonetheless $p$-vectors for which no realization exists. But the equation gives the possibility to add in some more polygons which act flat - they give in sum no positive or negative curvature. So while some sequences are not realizable, maybe it is possible to add some more polygons with no total curvature and try realize this sequence. A theorem which states that this will is always possible (where the exact meaning of ``this'' will be defined later) will be called a general Eberhard's theorem. This thesis proves some of these theorems positively and also gives some examples where adding certain polygons does not help in the realization. It contains three sections. The first specifies the conditions and equations stated above and introduces notation and the arguments used. The next five sections are each devoted to a specific general Eberhard theorem and prove them fully. The last section gives conditions where one can not hope for a general Eberhard's theorem and gives insight to some extensions to the present work. Each proof uses one or more figures to describe a construction. The general reading on most of them is ``the image shown left is transformed to the one on the right'' if the image is about to describe a specific transformation.
