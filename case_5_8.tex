\section{The $3$-valent case $[2 \times 5, 8]$}
\begin{construction}\label{thm:construction:5:8}
  Given a $3$-valent polyhedron $P$ with $p$-vector $(p_3, p_4, p_5, p_6, p_7, p_8, \dots, p_m)$ and $a \leq p_6$, there exists another $3$-valent polyhedron $P'$ with $p$-vector $(p_3, p_4, p'_5, a, p_7, p'_8, \dots, p_m)$, where $p'_5 - p_5 = 2(p'_8 - p_8)$.
  \begin{proof}
    As seen in \autoref{fig:case:5:8:img1}, one can create a $(2, 1, 2, 1)$-expansion of each polygon of the given realization $P$. The construction forms a ring of pentagons around these n-gons, afterwards another ring of octagons is added. The resulting figure has $n$ vertices of valence $3$ on its outer face, these will be the positions for the final set of $n$ pentagons resulting in the shape given in \autoref{fig:case:5:8:img1}.
    \begin{tikzfigure}{\label{fig:case:5:8:img1}}{A patch realizing the $(2, 1, 2, 1)$-expansion of a square}
      \matrix (m) [ column sep=1cm] {
        \begin{scope}
          \fill[fill=gray!50!white] (-1, -1) -- (1, -1) -- (1, 1) -- (-1, 1);
          \draw (-1, -1) circle[radius=2pt] -- (1, -1) circle[radius=2pt] -- (1, 1) circle[radius=2pt] -- (-1, 1) circle[radius=2pt] -- (-1, -1);
        \end{scope}
        &
        \begin{scope}[scale=0.25]
          \draw (-6, 0) -- (-7, -3) -- (-6, -6) -- (-3, -7) -- (0, -6) -- (3, -7) -- (6, -6) -- (7, -3) -- (6, 0) -- (7, 3) -- (6, 6) -- (3, 7) -- (0, 6) -- (-3, 7) -- (-6, 6) -- (-7, 3) -- (-6, 0);
          \draw (-5, 0) -- (-6, 0);
          \draw (0, -5) -- (0, -6);
          \draw (5, 0) -- (6, 0);
          \draw (0, 5) -- (0, 6);

          \draw (-5, 0) -- (-2, -2) -- (0, -5) -- (2, -2) -- (5, 0) -- (2, 2) -- (0, 5) -- (-2, 2) -- (-5, 0);
          \draw (-2, -2) -- (-1, -1);
          \draw (2, -2) -- (1, -1);
          \draw (2, 2) -- (1, 1);
          \draw (-2, 2) -- (-1, 1);

          \filldraw[fill=gray!50!white] (-1, -1) -- (1, -1) -- (1, 1) -- (-1, 1) -- (-1, -1);
          \draw (-7, -3)  -- (-9, -1) -- (-9, 1) circle[radius=8pt]-- (-7, 3);
          \draw ( 7, -3) -- ( 9, -1) circle[radius=8pt]-- ( 9, 1) -- ( 7, 3) ;
          \draw (-3, -7) -- (-1, -9) circle[radius=8pt]-- (1, -9) -- (3, -7) ;
          \draw (3, 7) -- (1, 9) circle[radius=8pt]-- (-1, 9) -- (-3, 7);

        \end{scope}
        \\
      };
    \end{tikzfigure}

    To achieve the requirement about the number of hexagons one can now replace the figure held in the first construction step with the ones seen in \autoref{fig:case:5:8:img2}. Note that the left side of the figure is the result of the above construction in the case of a hexagon and the right side has the same boundary structure, but consists only of pentagons and octagons. So one can substitute arbitrarily many hexagons by an amount of octagons and heptagons. The construction replaces each face with its $(2, 1, 2, 1)$-expansion, and since $(2, 1, 2, 1)$ is self-fitting, the construction gives rise to a new polyhedron $P'$  by \autoref{thm:construction:patch}. $p'_5 - p_5 = 2(p'_8 - p_8)$ follows, since \autoref{eq:valence:3} holds for $P$ as well as for $P'$, if $p'$ is the $p$-vector of $P'$.

    \begin{tikzfigure}{\label{fig:case:5:8:img2}}{The replacement figure in the case of a hexagon}
      \matrix (m) [ column sep=1cm] {

        \begin{scope}[rotate=-30, yscale=0.866, scale=0.25]
          \draw (-1, 8) -- (1, 8) -- (1.5, 7) -- (3.5, 7) -- (4.5, 5) -- (5.5, 5) -- (6.5, 3) -- (6, 2) -- (7, 0) -- (6, -2) -- (6.5, -3) -- (5.5, -5) -- (4.5, -5) -- (3.5, -7) -- (1.5, -7) -- (1, -8) -- (-1, -8) -- (-1.5, -7) -- (-3.5, -7) -- (-4.5, -5) -- (-5.5, -5) -- (-6.5, -3) -- (-6, -2) -- (-7, 0) -- (-6, 2) -- (-6.5, 3) -- (-5.5, 5) -- (-4.5, 5) -- (-3.5, 7) -- (-1.5, 7) -- (-1, 8);
          \draw (-1.5, 7) -- (0, 6) -- (1.5, 7);
          \draw (4.5, 5) -- (4.5, 3) -- (6, 2);
          \draw (4.5, -5) -- (4.5, -3) -- (6, -2);
          \draw (-1.5, -7) -- (0, -6) -- (1.5, -7);
          \draw (-4.5, 5) -- (-4.5, 3) -- (-6, 2);
          \draw (-4.5, -5) -- (-4.5, -3) -- (-6, -2);

          \draw (0, 6) -- (0, 4);
          \draw (4.5, 3) -- (3, 2);
          \draw (4.5, -3) -- (3, -2);
          \draw (0, -6) -- (0, -4);
          \draw (-4.5, 3) -- (-3, 2);
          \draw (-4.5, -3) -- (-3, -2);

          \draw (0, 4) -- (1, 2) -- (3, 2) -- (2, 0) -- (3, -2) -- (1, -2) -- (0, -4) -- (-1, -2) -- (-3, -2) -- (-2, 0) -- (-3, 2) -- (-1, 2) -- (0, 4);

          \draw (2, 0) -- (1, 0);
          \draw (1, 2) -- (0.5, 1);
          \draw (1, -2) -- (0.5, -1);
          \draw (-2, 0) -- (-1, 0);
          \draw (-1, 2) -- (-0.5, 1);
          \draw (-1, -2) -- (-0.5, -1);

          \filldraw[fill=gray!50!white] (1, 0) -- (0.5, 1) -- (-0.5, 1) -- (-1, 0) -- (-0.5, -1) -- (0.5, -1) -- (1, 0);

        \end{scope}
        &
        \begin{scope}[rotate=-30, yscale=0.866, scale=0.25]
          \draw (-1, 8) -- (1, 8) -- (1.5, 7) -- (3.5, 7) -- (4.5, 5) -- (5.5, 5) -- (6.5, 3) -- (6, 2) -- (7, 0) -- (6, -2) -- (6.5, -3) -- (5.5, -5) -- (4.5, -5) -- (3.5, -7) -- (1.5, -7) -- (1, -8) -- (-1, -8) -- (-1.5, -7) -- (-3.5, -7) -- (-4.5, -5) -- (-5.5, -5) -- (-6.5, -3) -- (-6, -2) -- (-7, 0) -- (-6, 2) -- (-6.5, 3) -- (-5.5, 5) -- (-4.5, 5) -- (-3.5, 7) -- (-1.5, 7) -- (-1, 8);

          \draw (-1.5, 7) -- (-1.5, 5) -- (1, 2) -- (2, 2) -- (2.5, 3) -- (1.5, 7);
          \draw (-6, 2) -- (-3.5, -1) -- (-2.5, -1) -- (-2, 0) -- (-3, 4) -- (-4.5, 5);
          \draw (4.5, -5) -- (4.5, -3) -- (2, -2) -- (1, -2) -- (0.5, -3) -- (0, -6) -- (1.5, -7);
          \draw (-3, 4) -- (-1.5, 5);
          \draw (-2, 0) -- (0, 0) -- (1, 2);
          \draw (0, 0) -- (1, -2);
          \draw (-2.5, -1) -- (0.5, -3);
          \draw (2, -2) -- (2, 2);
          \draw (-4.5, -5) -- (-4.5, -3) -- (-6, -2);
          \draw (-4.5, -3) -- (-3.5, -1);
          \draw (6, 2) -- (4.5, 3) -- (4.5, 5);
          \draw (4.5, 3) -- (2.5, 3);
          \draw (-1.5, -7) -- (0, -6);
          \draw (6, -2) -- (4.5, -3);
        \end{scope}

        \\
      };
    \end{tikzfigure}

    % \begin{figure}[htpp]
    %   \centering
    %   \begin{tikzpicture}
    %     \matrix (m) [ column sep=1cm] {
    %       \begin{scope}[xscale=1.0, yscale=0.866]
    %         \filldraw[fill=gray!50!white] (0, 1) -- ++(0.5, -1) -- ++(1, 0) -- ++(0.5, 1) -- ++(-0.5, 1) -- ++(-1, 0) -- ++(-0.5, -1);
    %         \filldraw[fill=gray!50!white] (1.5, 0) -- ++(0.5, -1) -- ++(1, 0) -- ++(0.5, 1) -- ++(-0.5, 1) -- ++(-1, 0) -- ++(-0.5, -1);
    %         \filldraw[fill=gray!50!white] (1.5, 2) -- ++(0.5, -1) -- ++(1, 0) -- ++(0.5, 1) -- ++(-0.5, 1) -- ++(-1, 0) -- ++(-0.5, -1);
    %       \end{scope}
    %       &


    %       \begin{scope}[rotate=40, xscale=1.0, yscale=0.866, scale=0.5]
    %         \filldraw[fill=gray!50!white] (0, 1) -- ++(0.5, -1) -- ++(1, 0) -- ++(0.5, 1) -- ++(-0.5, 1) -- ++(-1, 0) -- ++(-0.5, -1);
    %         \filldraw[fill=white] (-1.5, 0) -- ++(0.5, -1) -- ++(1, 0) -- ++(0.5, 1) -- ++(-0.5, 1) -- ++(-1, 0) -- ++(-0.5, -1);
    %         \filldraw[fill=white] (0, -1) -- ++(0.5, -1) -- ++(1, 0) -- ++(0.5, 1) -- ++(-0.5, 1) -- ++(-1, 0) -- ++(-0.5, -1);
    %         \filldraw[fill=white] (1.5, 0) -- ++(0.5, -1) -- ++(1, 0) -- ++(0.5, 1) -- ++(-0.5, 1) -- ++(-1, 0) -- ++(-0.5, -1);
    %         \filldraw[fill=white] (1.5, 2) -- ++(0.5, -1) -- ++(1, 0) -- ++(0.5, 1) -- ++(-0.5, 1) -- ++(-1, 0) -- ++(-0.5, -1);
    %         \filldraw[fill=white] (0, 3) -- ++(0.5, -1) -- ++(1, 0) -- ++(0.5, 1) -- ++(-0.5, 1) -- ++(-1, 0) -- ++(-0.5, -1);
    %         \filldraw[fill=white] (-1.5, 2) -- ++(0.5, -1) -- ++(1, 0) -- ++(0.5, 1) -- ++(-0.5, 1) -- ++(-1, 0) -- ++(-0.5, -1);

    %         \filldraw[fill=gray!50!white] (4.5, 0) -- ++(0.5, -1) -- ++(1, 0) -- ++(0.5, 1) -- ++(-0.5, 1) -- ++(-1, 0) -- ++(-0.5, -1);
    %         \filldraw[fill=white] (3, -1) -- ++(0.5, -1) -- ++(1, 0) -- ++(0.5, 1) -- ++(-0.5, 1) -- ++(-1, 0) -- ++(-0.5, -1);
    %         \filldraw[fill=white] (4.5, -2) -- ++(0.5, -1) -- ++(1, 0) -- ++(0.5, 1) -- ++(-0.5, 1) -- ++(-1, 0) -- ++(-0.5, -1);
    %         \filldraw[fill=white] (6, -1) -- ++(0.5, -1) -- ++(1, 0) -- ++(0.5, 1) -- ++(-0.5, 1) -- ++(-1, 0) -- ++(-0.5, -1);
    %         \filldraw[fill=white] (6, 1) -- ++(0.5, -1) -- ++(1, 0) -- ++(0.5, 1) -- ++(-0.5, 1) -- ++(-1, 0) -- ++(-0.5, -1);
    %         \filldraw[fill=white] (4.5, 2) -- ++(0.5, -1) -- ++(1, 0) -- ++(0.5, 1) -- ++(-0.5, 1) -- ++(-1, 0) -- ++(-0.5, -1);
    %         \filldraw[fill=white] (3, 1) -- ++(0.5, -1) -- ++(1, 0) -- ++(0.5, 1) -- ++(-0.5, 1) -- ++(-1, 0) -- ++(-0.5, -1);

    %         \filldraw[fill=gray!50!white] (1.5, -4) -- ++(0.5, -1) -- ++(1, 0) -- ++(0.5, 1) -- ++(-0.5, 1) -- ++(-1, 0) -- ++(-0.5, -1);
    %         \filldraw[fill=white] (0, -5) -- ++(0.5, -1) -- ++(1, 0) -- ++(0.5, 1) -- ++(-0.5, 1) -- ++(-1, 0) -- ++(-0.5, -1);
    %         \filldraw[fill=white] (1.5, -6) -- ++(0.5, -1) -- ++(1, 0) -- ++(0.5, 1) -- ++(-0.5, 1) -- ++(-1, 0) -- ++(-0.5, -1);
    %         \filldraw[fill=white] (3, -5) -- ++(0.5, -1) -- ++(1, 0) -- ++(0.5, 1) -- ++(-0.5, 1) -- ++(-1, 0) -- ++(-0.5, -1);
    %         \filldraw[fill=white] (3, -3) -- ++(0.5, -1) -- ++(1, 0) -- ++(0.5, 1) -- ++(-0.5, 1) -- ++(-1, 0) -- ++(-0.5, -1);
    %         \filldraw[fill=white] (1.5, -2) -- ++(0.5, -1) -- ++(1, 0) -- ++(0.5, 1) -- ++(-0.5, 1) -- ++(-1, 0) -- ++(-0.5, -1);
    %         \filldraw[fill=white] (0, -3) -- ++(0.5, -1) -- ++(1, 0) -- ++(0.5, 1) -- ++(-0.5, 1) -- ++(-1, 0) -- ++(-0.5, -1);
    %       \end{scope};
    %       \\
    %     };
    %   \end{tikzpicture}
    % \end{figure}


    % One now chooses a set $S$ of hexagons one wants to keep, $|S| = p_6$. Each hexagon not in $S$ and the surrounding six hexagons are replaced by the following structure:

    % \begin{figure}[htpp]
    %   \centering
    %   \begin{tikzpicture}
    %     \matrix (m) [ column sep=1cm] {
    %       \begin{scope}[xscale=1.0, yscale=0.866, scale=0.5]
    %         \draw (0, 1) -- ++(0.5, -1) -- ++(1, 0) -- ++(0.5, 1) -- ++(-0.5, 1) -- ++(-1, 0) -- ++(-0.5, -1);
    %         \draw (-1.5, 0) -- ++(0.5, -1) -- ++(1, 0) -- ++(0.5, 1) -- ++(-0.5, 1) -- ++(-1, 0) -- ++(-0.5, -1);
    %         \draw (0, -1) -- ++(0.5, -1) -- ++(1, 0) -- ++(0.5, 1) -- ++(-0.5, 1) -- ++(-1, 0) -- ++(-0.5, -1);
    %         \draw (1.5, 0) -- ++(0.5, -1) -- ++(1, 0) -- ++(0.5, 1) -- ++(-0.5, 1) -- ++(-1, 0) -- ++(-0.5, -1);
    %         \draw (1.5, 2) -- ++(0.5, -1) -- ++(1, 0) -- ++(0.5, 1) -- ++(-0.5, 1) -- ++(-1, 0) -- ++(-0.5, -1);
    %         \draw (0, 3) -- ++(0.5, -1) -- ++(1, 0) -- ++(0.5, 1) -- ++(-0.5, 1) -- ++(-1, 0) -- ++(-0.5, -1);
    %         \draw (-1.5, 2) -- ++(0.5, -1) -- ++(1, 0) -- ++(0.5, 1) -- ++(-0.5, 1) -- ++(-1, 0) -- ++(-0.5, -1);
    %       \end{scope};

    %       &

    %       \begin{scope}[xscale=1.0, yscale=0.866, scale=0.5]
    %         \draw (-1.5, 0) -- ++(0.5, -1) -- ++(1, 0) -- ++(0.25, 1.5) -- ++(-1.25, 0.5) -- ++(-0.5, -1);
    %         \draw (0, -1) -- ++(0.5, -1) -- ++(1, 0) -- ++(0.5, 1) -- ++(-0.25, 1.5) -- ++(-0.75, 0.5) -- ++(-0.75, -0.5);
    %         \draw (1.75, 0.5) -- ++(0.25, -1.5) -- ++(1, 0) -- ++(0.5, 1) -- ++(-0.5, 1) -- ++(-1.25, -0.5);
    %         \draw (1, 1) -- ++(0.75, -0.5) -- ++(1.25, 0.5) -- ++(0.5, 1) -- ++(-0.5, 1) -- ++(-1, 0) -- ++(-1, -1);
    %         \draw (0, 3) -- ++(1, -1) -- ++(1, 1) -- ++(-0.5, 1) -- ++(-1, 0) -- ++(-0.5, -1);
    %         \draw (-1.5, 2) -- ++(0.5, -1) -- ++(1.25, -0.5) -- ++(0.75, 0.5) -- ++(0, 1) -- ++(-1, 1) -- ++(-1, 0) -- ++(-0.5, -1);
    %       \end{scope};

    %       \\
    %     };
    %   \end{tikzpicture}
    % \end{figure}

    % For the remaining hexagons and all other kinds of polygons this ring structure is replaced differently, for each $n$-gon $n$ pentagons and $n$ heptagons replace the ring of hexagons:

    % \begin{figure}[htpp]
    %   \centering
    %   \begin{tikzpicture}
    %     \matrix (m) [ column sep=1cm] {

    %       \begin{scope}[scale=0.25]
    %         \draw (-3, 0) -- (0, -3) -- (3, 0) -- (0, 3) -- (-3, 0);
    %         \draw (-3, 0) -- (-6, 0);
    %         \draw (0, -3) -- (0, -6);
    %         \draw (3, 0) -- (6, 0);
    %         \draw (0, 3) -- (0, 6);
    %         \draw (-6, 0) -- (-7, -3) -- (-3, -7) -- (0, -6) -- (3, -7) -- (7, -3) -- (6, 0) -- (7, 3) -- (3, 7) -- (0, 6) -- (-3, 7) -- (-7, 3) -- (-6, 0);
    %       \end{scope}
    %       &
    %       \begin{scope}[scale=0.25]
    %         \draw (-6, 0) -- (-7, -3) -- (-6, -6) -- (-3, -7) -- (0, -6) -- (3, -7) -- (6, -6) -- (7, -3) -- (6, 0) -- (7, 3) -- (6, 6) -- (3, 7) -- (0, 6) -- (-3, 7) -- (-6, 6) -- (-7, 3) -- (-6, 0);
    %         \draw (-5, 0) -- (-6, 0);
    %         \draw (0, -5) -- (0, -6);
    %         \draw (5, 0) -- (6, 0);
    %         \draw (0, 5) -- (0, 6);

    %         \draw (-5, 0) -- (-2, -2) -- (0, -5) -- (2, -2) -- (5, 0) -- (2, 2) -- (0, 5) -- (-2, 2) -- (-5, 0);
    %         \draw (-2, -2) -- (-1, -1);
    %         \draw (2, -2) -- (1, -1);
    %         \draw (2, 2) -- (1, 1);
    %         \draw (-2, 2) -- (-1, 1);

    %         \draw (-1, -1) -- (1, -1) -- (1, 1) -- (-1, 1) -- (-1, -1);
    %         \draw (-7, -3) -- (-9, -1) -- (-9, 1) -- (-7, 3);
    %         \draw ( 7, -3) -- ( 9, -1) -- ( 9, 1) -- ( 7, 3);
    %         \draw (-3, -7) -- (-1, -9) -- (1, -9) -- (3, -7);
    %         \draw (3, 7) -- (1, 9) -- (-1, 9) -- (-3, 7);

    %       \end{scope}

    %       \\
    %     };
    %   \end{tikzpicture}
    % \end{figure}
  \end{proof}
\end{construction}

\begin{corollary}
  Let $p = (p_3, p_4, p_5, \dots, p_n)$ be a given sequence satisfying \autoref{eq:valence:3}, then $p$ is $[2 \times 5, 8]$-$3$-realizable.
  \begin{proof}
    The proof starts by using the construction given by Eberhard's \autoref{thm:eberhard:3}. This results in a realization of $p$ which has many more hexagons then needed. The next step is to remove these hexagons and replace them by pentagons and heptagons. This is done by \autoref{thm:construction:5:8}. The resulting polyhedron has the required amount of polygons of each shape.
  \end{proof}
\end{corollary}
