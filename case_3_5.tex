\section{The $4$-valent case $[3, 5]$}
In this section the main problem will be proven for the case of adding the same number of triangles and pentagons to realize a given sequence satisfying \autoref{eq:valence:4}. This proof will be given by an explicit construction consisting of several building blocks which are described below.

\begin{definition}[Square patch]\label{def:square:patch}
  In the case of $4$-valent polyhedrons a square patch denotes a $4$-patch where exactly four vertices on the outer have face  count $1$, while every other boundary vertex has face count $2$.
\end{definition}

%% \begin{lemma}\label{thm:square:arc}
%%   Let $G$ be $4$-component with an square outer arc, consisting of $p_3$ triangles, $p_4$ quadrangles and so forth. Then $p$ fullfills \autoref{eq:zero:curv:4}.
%%   \begin{proof}
%%     Let $v_4$ be the number of inner vertices, $v_3$ the number of vertices on the outer arc with face count $2$ and $v_2$ the number of vertices with face count $1$. Note that $v_i$ counts the number of vertices of valence $i$. Let $v = v_2 + v_3 + v_4$ be the number of vertices, $e$ be the number of edges and $f$ the number of faces of $G$. The outer face is circumscribed by the outer arc and is therefore a $v_2+v_3$-gon. Then by \autoref{def:square:arc} and a double counting argument follows:
%%     \begin{align*}
%%       v + f - e &= 2 \\
%%       v_2 &= 4 \\
%%       f &= \sum_{k \geq 3} p_k + 1 \\
%%       e &= \frac{1}{2}(4 v_4 + 3 v_3  + 2 v_2) \\
%%       e &= \frac{1}{2}\left(\sum_{k \geq 3} k p_k + v_2 + v_3\right).
%%     \end{align*}
%%     Plugging these into \autoref{thm:eulers:relation} $v + f - e = 2$ results in
%%     \begin{align*}
%%       4 = 2(v + f - e)&= 2(v_2 + v_3 + v_4) + 2\left(\sum_{k \geq 3} p_k + 1\right) - \frac{1}{2}\left(\sum_{k \geq 3} k p_k + v_2 + v_3\right) - \frac{1}{2}(4 v_4 + 3 v_3  + 2 v_2)\\
%%       &= \left((2v_2 - \frac{1}{2}v_2 - v_2)\right) + \left((2v_3 - \frac{1}{2}v_3 - \frac{3}{2}v_3)\right) + (2v_4 - 2v_4) + \frac{1}{2} \left(\sum_{k \geq 3} (4 - k) p_k \right) + 2, \\
%%       &= 4 + \frac{1}{2} \left(\sum_{k \geq 3} (4 - k) p_k \right)
%%     \end{align*}
%%     resulting in \autoref{eq:zero:curv:4} when multiplying with $2$.
%%   \end{proof}
%% \end{lemma}

\begin{definition}[$2 \times 2$ building block]
  The following square patch consisting of four triangles and four pentagons will be called a $2 \times 2$ building block.
  \begin{tikzfigure}{\label{fig:case:3:5:2times2bb}}{$2 \times 2$ building block}
    \begin{scope}
      \fill[fill=gray!50!white] (0, 0.5) -- (0, 1) -- (1, 1) -- (1, 0) -- (0.5, 0);
      \draw (-1, -1) -- (1, -1) -- (1, 1) -- (-1, 1) -- (-1, -1);
      \draw (-0.5, 0) -- (0, -0.5) -- (0.5, 0) -- (0, 0.5) -- (-0.5, 0);
      \draw (0, -1) -- (0, 1);
      \draw (-1, 0) -- (1, 0);
    \end{scope}
  \end{tikzfigure}
\end{definition}

Each $k$-gon, $(k > 5)$ can be grouped together with $k-4$ triangles to create a ``flat'' block (flat meaning they satisfy \autoref{eq:zero:curv:4}). This can be further transformed into a square patch by adding $2 \times 2$ building blocks, as well as some additional triangles and pentagons, as seen in \autoref{fig:case:3:5:basicbb}.
\begin{definition}[Basic building block] For a $k$-gon with $k > 5$, the $4$-patch in \autoref{fig:case:3:5:basicbb} will be called its ``basic building block''.
  \begin{tikzfigure}{\label{fig:case:3:5:basicbb}}{Basic building block for a $k$-gon, $k>5$, here a nonagon}
    \begin{scope}[scale=0.1]
      \fill[fill=gray!50!white] (-10, 10) -- (6, 10) -- (10, 6) -- (10, -10) -- (6, -10) -- (-10, 6) -- cycle;
      \draw (-10, -10) -- (10, -10) -- (10, 10) -- (-10, 10) -- (-10, -10);
      \draw (6, 10) -- (10, 6);
      \draw (6, -10) -- (-10, 6);
      \draw (2, -10) -- (2, -6) -- (-10, -6);
      \draw (-2, -10) -- (-2, -2) -- (-10, -2);
      \draw (-6, -10) -- (-6, 2) -- (-10, 2);
      \draw (2, -10) -- (2, -6) -- (-10, -6);
      \draw (-2, -10) -- (-2, -2) -- (-10, -2);
      \draw (-6, -10) -- (-6, 2) -- (-10, 2);
      \draw (-10, 0) -- (-8, -2) -- (-8, -10);
      \draw (-10, -4) -- (-6, -4) -- (-4, -6) -- (-4, -10);
      \draw (-10, -8) -- (-2, -8) -- (0, -10);
      \draw (-3, -8) -- (-4, -9) -- (-5, -8) -- (-4, -7) -- cycle;
      \draw (-8, -3) -- (-9, -4) -- (-8, -5) -- (-7, -4) -- cycle;
      \draw (-8, -7) -- (-9, -8) -- (-8, -9) -- (-7, -8) -- cycle;
    \end{scope}
  \end{tikzfigure}
  
  For $k = 4$, a quadrangle, the structures in \autoref{fig:case:3:5:quadbb} will be considered as the respective basic building block. For a pentagon the basic building block is seen in \autoref{fig:case:3:5:2times2bb}.

  \begin{tikzfigure}{\label{fig:case:3:5:quadbb}}{Basic building block for a quadrangle}
    \begin{scope}[scale=0.5]
      \fill[fill=gray!50!white] (1, 1) -- (2, 1) -- (2, 2) -- (1, 2);
      \draw (-2, -2) -- (-2, 2) -- (2, 2) -- (2, -2) -- cycle;
      \draw (1, -2) -- (1, 2);
      \draw (-2, 1) -- (2, 1);
      \draw (-2, 0) -- (-1, -1) -- (0, 0) -- (1, -1) -- (2, 0);
      \draw (-2, -1) -- (-1, 0) -- (0, -1) -- (1, 0) -- (2, -1);
      \draw (-1, -2) -- (-1, 1) -- (0, 2);
      \draw (0, -2) -- (0, 1) -- (-1, 2);
    \end{scope}
  \end{tikzfigure}

  Note that every basic building block is a square patch, where the north and east as well as the south and west ``sides'' have the same amount of edges, which is even. Each building block consists of the assigned $k$-gon, $k-4+r$ triangles and $r$ pentagons for some value $r \in \nats$.
\end{definition}

\begin{definition}[Basis] One is able to assemble many basic building blocks by stacking them diagonally as seen in \autoref{fig:case:3:5:basis} for the basic building blocks of a quadrangle, and pentagon and a nonagon. This shape can be completed to a square patch by inserting $(2 \times 2)$ building blocks. This is possible since each side of a building block has an even number of edges.
  \begin{tikzfigure}{\label{fig:case:3:5:basis}}{The assembly of the basis from basic building blocks}
    \begin{scope}[scale=0.1]
      \fill[fill=gray!50!white] (-15, -15) -- (-10, -15) -- (-10, -10) -- (-15, -10);
      \draw[very thick] (-30, -30) -- (-30, -10) -- (-10, -10) -- (-10, -30) -- cycle;
      \draw (-15, -30) -- (-15, -10);
      \draw (-30, -15) -- (-10, -15);
      \draw (-30, -20) -- (-25, -25) -- (-20, -20) -- (-15, -25) -- (-10, -20);
      \draw (-30, -25) -- (-25, -20) -- (-20, -25) -- (-15, -20) -- (-10, -25);
      \draw (-25, -30) -- (-25, -15) -- (-20, -10);
      \draw (-20, -30) -- (-20, -15) -- (-25, -10);
      
      \fill[fill=gray!50!white] (-10, 10) -- (6, 10) -- (10, 6) -- (10, -10) -- (6, -10) -- (-10, 6) -- cycle;
      \draw[very thick] (-10, -10) -- (10, -10) -- (10, 10) -- (-10, 10) -- (-10, -10);
      \draw (6, 10) -- (10, 6);
      \draw (6, -10) -- (-10, 6);
      \draw (2, -10) -- (2, -6) -- (-10, -6);
      \draw (-2, -10) -- (-2, -2) -- (-10, -2);
      \draw (-6, -10) -- (-6, 2) -- (-10, 2);
      \draw (2, -10) -- (2, -6) -- (-10, -6);
      \draw (-2, -10) -- (-2, -2) -- (-10, -2);
      \draw (-6, -10) -- (-6, 2) -- (-10, 2);
      \draw (-10, 0) -- (-8, -2) -- (-8, -10);
      \draw (-10, -4) -- (-6, -4) -- (-4, -6) -- (-4, -10);
      \draw (-10, -8) -- (-2, -8) -- (0, -10);
      \draw (-3, -8) -- (-4, -9) -- (-5, -8) -- (-4, -7) -- cycle;
      \draw (-8, -3) -- (-9, -4) -- (-8, -5) -- (-7, -4) -- cycle;
      \draw (-8, -7) -- (-9, -8) -- (-8, -9) -- (-7, -8) -- cycle;

      \fill[fill=gray!50!white] (20, 25) -- (20, 30) -- (30, 30) -- (30, 20) -- (25, 20);
      \draw[very thick] (10, 10) -- (30, 10) -- (30, 30) -- (10, 30) -- cycle;
      \draw (15, 20) -- (20, 15) -- (25, 20) -- (20, 25) -- (15, 20);
      \draw (20, 10) -- (20, 30);
      \draw (10, 20) -- (30, 20);


      \draw ( 10,  6) -- ( 30,  6);
      \draw ( 10,-10) -- ( 30,-10);
      \draw (-10,-15) -- ( 30,-15);
      \draw (-10,-20) -- ( 30,-20);
      \draw (-10,-25) -- ( 30,-25);
      \draw (-10,-30) -- ( 30,-30);

      \draw ( -8,-30) -- ( -8,-10);
      \draw ( -6,-30) -- ( -6,-10);
      \draw ( -4,-30) -- ( -4,-10);
      \draw ( -2,-30) -- ( -2,-10);
      \draw (  0,-30) -- (  0,-10);
      \draw (  2,-30) -- (  2,-10);
      \draw (  6,-30) -- (  6,-10);
      \draw ( 10,-30) -- ( 10,-10);
      \draw ( 20,-30) -- ( 20, 10);
      \draw ( 30,-30) -- ( 30, 10);
      
      \draw ( 18,  6) -- ( 20,  4) -- ( 22,  6) -- ( 20,  8) -- cycle;
      \draw ( 18,-15) -- ( 20,-17) -- ( 22,-15) -- ( 20,-13) -- cycle;
      \draw ( 18,-25) -- ( 20,-27) -- ( 22,-25) -- ( 20,-23) -- cycle;
      \draw ( -9,-15) -- ( -8,-16) -- ( -7,-15) -- ( -8,-14) -- cycle;
      \draw ( -5,-15) -- ( -4,-16) -- ( -3,-15) -- ( -4,-14) -- cycle;
      \draw ( -1,-15) -- (  0,-16) -- (  1,-15) -- (  0,-14) -- cycle;
      \draw (  4,-15) -- (  6,-17) -- (  8,-15) -- (  6,-13) -- cycle;
      \draw ( -9,-25) -- ( -8,-26) -- ( -7,-25) -- ( -8,-24) -- cycle;
      \draw ( -5,-25) -- ( -4,-26) -- ( -3,-25) -- ( -4,-24) -- cycle;
      \draw ( -1,-25) -- (  0,-26) -- (  1,-25) -- (  0,-24) -- cycle;
      \draw (  4,-25) -- (  6,-27) -- (  8,-25) -- (  6,-23) -- cycle;

      \draw (  6, 10) -- (  6, 30);
      \draw (-10, 10) -- (-10, 30);
      \draw (-15,-10) -- (-15, 30);
      \draw (-20,-10) -- (-20, 30);
      \draw (-25,-10) -- (-25, 30);
      \draw (-30,-10) -- (-30, 30);

      \draw (-30, -8) -- (-10, -8);
      \draw (-30, -6) -- (-10, -6);
      \draw (-30, -4) -- (-10, -4);
      \draw (-30, -2) -- (-10, -2);
      \draw (-30,  0) -- (-10,  0);
      \draw (-30,  2) -- (-10,  2);
      \draw (-30,  6) -- (-10,  6);
      \draw (-30, 10) -- (-10, 10);
      \draw (-30, 20) -- ( 10, 20);
      \draw (-30, 30) -- ( 10, 30);
      
      \draw (  6, 18) -- (  4, 20) -- (  6, 22) -- (  8, 20) -- cycle;
      \draw (-15, 18) -- (-17, 20) -- (-15, 22) -- (-13, 20) -- cycle;
      \draw (-25, 18) -- (-27, 20) -- (-25, 22) -- (-23, 20) -- cycle;
      \draw (-15, -9) -- (-16, -8) -- (-15, -7) -- (-14, -8) -- cycle;
      \draw (-15, -5) -- (-16, -4) -- (-15, -3) -- (-14, -4) -- cycle;
      \draw (-15, -1) -- (-16,  0) -- (-15,  1) -- (-14,  0) -- cycle;
      \draw (-15,  4) -- (-17,  6) -- (-15,  8) -- (-13,  6) -- cycle;
      \draw (-25, -9) -- (-26, -8) -- (-25, -7) -- (-24, -8) -- cycle;
      \draw (-25, -5) -- (-26, -4) -- (-25, -3) -- (-24, -4) -- cycle;
      \draw (-25, -1) -- (-26,  0) -- (-25,  1) -- (-24,  0) -- cycle;
      \draw (-25,  4) -- (-27,  6) -- (-25,  8) -- (-23,  6) -- cycle;

    \end{scope}
  \end{tikzfigure}

  Similar to the basic building blocks, the north and east as well as the west and south sides share the same number of edges.
\end{definition}

\begin{theorem}\label{thm:case3:5} Let $p = (p_3, p_4, p_5, \dots, p_n)$ be a given sequence satisfying \autoref{eq:valence:4}. Then $p$ is $[3, 5]$-$4$-realizable.
  \begin{proof}
    For each $k$-gon, $k > 3$ take the basic building block for this $k$-gon and form a basis out of these. Let $p'_k$ be the number of $k$-gons for each $3 \geq k$. According to the construction, the basis only consists of the required $k$-gons, triangles and pentagons, so $p_k = p'_k$ for $k \neq 3, 5$. The basis can be closed in two steps. First add a ring of $2 \times 2$ building blocks around the base and put the pieces seen on the left of \autoref{fig:case3:5:closedbasis1} in each corner (here four of the additional triangles are used, one in each corner). The construction is possible, since every side of the base has a length divisible by two. The resulting figure has an outer face with twelve additional edges compared to the basis, so the number of edges is divisible by four. Also note, that each vertex on the outer face is adjacent to two faces.
    \begin{tikzfigure}{\label{fig:case3:5:closedbasis1}}{Part one of closing the basis}
      \matrix (m) [ column sep=1cm] {
        \begin{scope}[scale=0.8]
          \draw (2, 1) -- (-1, 1) -- (-1, 3) -- (1, 3) -- (2, 2) -- (2, 0) -- (0, 0) -- (0, 3);
          \draw (0, 1.5) -- (-0.5, 2) -- (0, 2.5) -- (0.5, 2) -- (0.5, 1) -- (1, 0.5) -- (1.5, 1) -- (1, 1.5) -- cycle;
          \draw (1, 0) -- (1, 3);
          \draw (-1, 2) -- (2, 2);
          \draw (0.25, 1.5) -- (0.5, 1.75) -- (0.75, 1.5) -- (0.5, 1.25) -- cycle;
        \end{scope}
        &
        \begin{scope}[scale=0.4]
          \draw (-5, -5) -- (5, -5) -- (5, 5) -- (-5, 5) -- cycle;

          \draw (7, 5) -- (4, 5) -- (4, 7) -- (6, 7) -- (7, 6) -- (7, 4) -- (5, 4) -- (5, 7);
          \draw (5, 5.5) -- (4.5, 6) -- (5, 6.5) -- (5.5, 6) -- (5.5, 5) -- (6, 4.5) -- (6.5, 5) -- (6, 5.5) -- cycle;
          \draw (6, 4) -- (6, 7);
          \draw (4, 6) -- (7, 6);
          \draw (5.25, 5.5) -- (5.5, 5.75) -- (5.75, 5.5) -- (5.5, 5.25) -- cycle;

          \draw (-7, 5) -- (-4, 5) -- (-4, 7) -- (-6, 7) -- (-7, 6) -- (-7, 4) -- (-5, 4) -- (-5, 7);
          \draw (-5, 5.5) -- (-4.5, 6) -- (-5, 6.5) -- (-5.5, 6) -- (-5.5, 5) -- (-6, 4.5) -- (-6.5, 5) -- (-6, 5.5) -- cycle;
          \draw (-6, 4) -- (-6, 7);
          \draw (-4, 6) -- (-7, 6);
          \draw (-5.25, 5.5) -- (-5.5, 5.75) -- (-5.75, 5.5) -- (-5.5, 5.25) -- cycle;

          \draw (-7, -5) -- (-4, -5) -- (-4, -7) -- (-6, -7) -- (-7, -6) -- (-7, -4) -- (-5, -4) -- (-5, -7);
          \draw (-5, -5.5) -- (-4.5, -6) -- (-5, -6.5) -- (-5.5, -6) -- (-5.5, -5) -- (-6, -4.5) -- (-6.5, -5) -- (-6, -5.5) -- cycle;
          \draw (-6, -4) -- (-6, -7);
          \draw (-4, -6) -- (-7, -6);
          \draw (-5.25, -5.5) -- (-5.5, -5.75) -- (-5.75, -5.5) -- (-5.5, -5.25) -- cycle;

          \draw (7, -5) -- (4, -5) -- (4, -7) -- (6, -7) -- (7, -6) -- (7, -4) -- (5, -4) -- (5, -7);
          \draw (5, -5.5) -- (4.5, -6) -- (5, -6.5) -- (5.5, -6) -- (5.5, -5) -- (6, -4.5) -- (6.5, -5) -- (6, -5.5) -- cycle;
          \draw (6, -4) -- (6, -7);
          \draw (4, -6) -- (7, -6);
          \draw (5.25, -5.5) -- (5.5, -5.75) -- (5.75, -5.5) -- (5.5, -5.25) -- cycle;

          \draw (6, 3) +(-1, -1) -- +(1, -1) -- +(1, 1) -- +(-1, 1) -- +(-1, -1) +(-1, 0) -- +(1, 0) +(0, -1) -- +(0, 1) +(0, -0.5) -- +(0.5, 0) -- +(0, 0.5) -- +(-0.5, 0) -- +(0, -0.5);
          \draw (3, 6) +(-1, -1) -- +(1, -1) -- +(1, 1) -- +(-1, 1) -- +(-1, -1) +(-1, 0) -- +(1, 0) +(0, -1) -- +(0, 1) +(0, -0.5) -- +(0.5, 0) -- +(0, 0.5) -- +(-0.5, 0) -- +(0, -0.5);

          \draw (-6, 3) +(-1, -1) -- +(1, -1) -- +(1, 1) -- +(-1, 1) -- +(-1, -1) +(-1, 0) -- +(1, 0) +(0, -1) -- +(0, 1) +(0, -0.5) -- +(0.5, 0) -- +(0, 0.5) -- +(-0.5, 0) -- +(0, -0.5);
          \draw (-3, 6) +(-1, -1) -- +(1, -1) -- +(1, 1) -- +(-1, 1) -- +(-1, -1) +(-1, 0) -- +(1, 0) +(0, -1) -- +(0, 1) +(0, -0.5) -- +(0.5, 0) -- +(0, 0.5) -- +(-0.5, 0) -- +(0, -0.5);

          \draw (6, -3) +(-1, -1) -- +(1, -1) -- +(1, 1) -- +(-1, 1) -- +(-1, -1) +(-1, 0) -- +(1, 0) +(0, -1) -- +(0, 1) +(0, -0.5) -- +(0.5, 0) -- +(0, 0.5) -- +(-0.5, 0) -- +(0, -0.5);
          \draw (3, -6) +(-1, -1) -- +(1, -1) -- +(1, 1) -- +(-1, 1) -- +(-1, -1) +(-1, 0) -- +(1, 0) +(0, -1) -- +(0, 1) +(0, -0.5) -- +(0.5, 0) -- +(0, 0.5) -- +(-0.5, 0) -- +(0, -0.5);

          \draw (-6, -3) +(-1, -1) -- +(1, -1) -- +(1, 1) -- +(-1, 1) -- +(-1, -1) +(-1, 0) -- +(1, 0) +(0, -1) -- +(0, 1) +(0, -0.5) -- +(0.5, 0) -- +(0, 0.5) -- +(-0.5, 0) -- +(0, -0.5);
          \draw (-3, -6) +(-1, -1) -- +(1, -1) -- +(1, 1) -- +(-1, 1) -- +(-1, -1) +(-1, 0) -- +(1, 0) +(0, -1) -- +(0, 1) +(0, -0.5) -- +(0.5, 0) -- +(0, 0.5) -- +(-0.5, 0) -- +(0, -0.5);

          \draw[loosely dotted] (6, 2) -- (6, -2);
          \draw[loosely dotted] (-6, 2) -- (-6, -2);
          \draw[loosely dotted] (-2, 6) -- (2, 6);
          \draw[loosely dotted] (-2, -6) -- (2, -6);
          
          
          \node at (0, 0) {Basis};
        \end{scope}
        \\
      };
    \end{tikzfigure}

    The second step uses the remaining four triangles and several $2 \times 2$ building blocks and creates a fitting stripe, see \autoref{fig:case3:5:closedbasis2}. By adding in $2 \times 2$-building blocks, one can create stripes, with number of edges on the outer face equal to any multiple of four - in this case the number of edges of the previous construction. As before, each vertex on the outer face is adjacent to two faces, therefore, after glueing the results of both steps together, the polyhedron held is $4$-valent.
    \begin{tikzfigure}{\label{fig:case3:5:closedbasis2}}{Part two of closing the basis}
      \begin{scope}[scale=0.8]
        \draw (-4, -1) -- (-5, 0) -- (-4, 1);
        \draw (4, -1) -- (5, 0) -- (4, 1);
        \draw (-4, 0) -- (-5, 0);
        \draw (4, 0) -- (5, 0);
        \draw (-3, 0) +(-1, -1) -- +(1, -1) -- +(1, 1) -- +(-1, 1) -- +(-1, -1) +(-1, 0) -- +(1, 0) +(0, -1) -- +(0, 1) +(0, -0.5) -- +(0.5, 0) -- +(0, 0.5) -- +(-0.5, 0) -- +(0, -0.5);
        \draw (3, 0) +(-1, -1) -- +(1, -1) -- +(1, 1) -- +(-1, 1) -- +(-1, -1) +(-1, 0) -- +(1, 0) +(0, -1) -- +(0, 1) +(0, -0.5) -- +(0.5, 0) -- +(0, 0.5) -- +(-0.5, 0) -- +(0, -0.5);
        \draw[loosely dotted] (-2, 1) -- (2, 1);
        \draw[loosely dotted] (-2, 0) -- (2, 0);
        \draw[loosely dotted] (-2, -1) -- (2, -1);
      \end{scope}
    \end{tikzfigure}
    For the $p$-vector $p''$ of the obtained polyhedron this gives
    \begin{align*}
      0 = 8 - 8 =& \sum_{k=3}^n (4 - k) p_k'' - \sum_{k=3}^n (4 - k) p_k\\
      \implies&  p''_3 - p_3 = p''_5 - p_5, 
    \end{align*}
    from \autoref{eq:valence:4} and since $p_k = p''_k$ for $k\neq 3, 5$. Thus $p'' = p + c [2 \times 3, 5]$ for some $c \in \nats$.
  \end{proof}
  \end{theorem}


%% \begin{tikzfigure}{\label{fig:case3:5:closedbasis1}}
%%       \begin{scope}
%%         %% \filldraw[fill=gray!50!white]
%%         %% (-1, 1) arc [start angle = 180, delta angle = -60, radius = 2]
%%         %% node(n1){} arc [start angle = 60, delta angle = 150, radius = 2]
%%         %% node(w1){} arc [start angle = 150, delta angle = -60, radius = 2];
        
%%         %% \filldraw[fill=gray!50!white]
%%         %% (1, -1) arc [start angle = 0, delta angle = -60, radius = 2]
%%         %% node(s1){} arc [start angle = 240, delta angle = 150, radius = 2]
%%         %% node(e1){} arc [start angle = 330, delta angle = -60, radius = 2];

        
        
%%         %% \fill[fill=gray!50!white] (1, 1) -- (1, 0.6) arc [start angle = 270, delta angle = 270, radius = 0.4]  -- cycle;
%%         %% \fill[fill=gray!50!white] (-1, -1) -- (-1, -0.6) arc [start angle = 90, delta angle = 270, radius = 0.4] -- cycle;
%%         \draw (-1, -1) -- (1, -1) -- (1, 1) -- (-1, 1) -- cycle;
%%         \node (0,0) {Basis};
%%         %% \path [name path=nw1] (1, 1) arc [start angle = 0, delta angle = 270, radius = 2];
%%         %% \path [name path=se1] (-1, -1) arc [start angle = 180, delta angle = 270, radius = 2];

%%         %% \draw [name path=ne2] (1, -0.6) arc [start angle = 270, delta angle = 270, radius = 1.6];
%%         %% \draw [name path=ne3] (1, 0.1) arc [start angle = 270, delta angle = 270, radius = 0.9];
%%         %% \draw [name path=ne4] (1, 0.3) arc [start angle = 270, delta angle = 270, radius = 0.7];
%%         %% \draw [name path=ne5] (1, 0.6) arc [start angle = 270, delta angle = 270, radius = 0.4];
        
%%         %% \draw [name path=sw2] (-1, 0.3) arc [start angle = 90, delta angle = 270, radius = 1.3];
%%         %% \draw [name path=sw3] (-1, 0.1) arc [start angle = 90, delta angle = 270, radius = 1.1];
%%         %% \draw [name path=sw4] (-1, -0.6) arc [start angle = 90, delta angle = 270, radius = 0.4];

%%         %% \path [name intersections={of=nw1 and ne2, by=n2}];
%%         %% \path [name intersections={of=nw1 and ne3, by=n3}];
%%         %% \path [name intersections={of=nw1 and ne4, by=n4}];
%%         %% \path [name intersections={of=nw1 and ne5, by=n5}];

%%         %% \path [name intersections={of=se1 and ne2, by=e2}];
%%         %% \path [name intersections={of=se1 and ne3, by=e3}];
%%         %% \path [name intersections={of=se1 and ne4, by=e4}];
%%         %% \path [name intersections={of=se1 and ne5, by=e5}];

%%         %% \path [name intersections={of=nw1 and sw2, by=w2}];
%%         %% \path [name intersections={of=nw1 and sw3, by=w3}];
%%         %% \path [name intersections={of=nw1 and sw4, by=w4}];

%%         %% \path [name intersections={of=se1 and sw2, by=s2}];
%%         %% \path [name intersections={of=se1 and sw3, by=s3}];
%%         %% \path [name intersections={of=se1 and sw4, by=s4}];

        
%%         %% %% \draw (n1) arc [start angle = 120, delta angle = -150, radius = 2];
%%         %% %% \draw (w1) arc [start angle = 150, delta angle = 150, radius = 2];

%%         %% \draw (1, 1) -- (n5) to [bend left=60] (e4) to [bend right=60] (n3) to [bend left=60] (e2);
%%         %% \draw (1, 1) -- (e5) to [bend right=60] (n4) to [bend left=60] (e3) to [bend right=60] (n2);

%%         \node (0,0) {Basis};

%%         %% \draw (1, -1) -- (3, 1) -- (1, 3) -- (-1, 1);
%%         %% \draw (1, -0.6) -- (2.6, 1) -- (1, 2.6) -- (-0.6, 1);
%%         %% \draw (1, 0.1) -- (1.9, 1) -- (1, 1.9) -- (0.1, 1);
%%         %% \draw (1, 0.3) -- (1.7, 1) -- (1, 1.7) -- (0.3, 1);
%%         %% \draw (1, 0.6) -- (1.4, 1) -- (1, 1.4) -- (0.6, 1);
%%         %% \draw (1, 1) -- (1.4, 1) -- (1, 1.7) -- (1.9, 1) -- (1, 2.6) -- (3, 1) -- (3, -3);
%%         %% \draw (1, 1) -- (1, 1.4) -- (1.7, 1) -- (1, 1.9) -- (2.6, 1) -- (1, 3) -- (-3, 3);

%%         %% \draw (-1, 1) -- (-3, -1) -- (-1, -3) -- (1, -1);
%%         %% \draw (-1, 0.3) -- (-2.3, -1) -- (-1, -2.3) -- (0.3, -1);
%%         %% \draw (-1, 0.1) -- (-2.1, -1) -- (-1, -2.1) -- (0.1, -1);
%%         %% \draw (-1, -0.6) -- (-1.4, -1) -- (-1, -1.4) -- (-0.6, -1);
%%         %% \draw (-1, -1) -- (-1.4, -1) -- (-1, -2.1) -- (-2.3, -1) -- (-1, -3) -- (3, -3);
%%         %% \draw (-1, -1) -- (-1, -1.4) -- (-2.1, -1) -- (-1, -2.3) -- (-3, -1) -- (-3, 3);
%%       \end{scope}
%%     \end{tikzfigure}
