\section{The $3$-valent case $[5, 7]$}
This theorem is already established in \cite{devos2010eberhard}. Here we present a much simpler proof (for a part of their work):


\begin{construction}\label{thm:construction:5:7}
  Given a $3$-valent polyhedron $P$ with $p$-vector $(p_3, p_4, p_5, p_6, p_7, p_8, \dots, p_m)$ and $a \leq p_6$, there exists $p'_5, p'_7 \in \nats, p'_5 \geq p_5, p_7' \geq p_7, p'_5 - p_5 = p'_7 - p_7$ and another $3$-valent polyhedron $P'$ with $p$-vector $(p_3, p_4, p'_5, a, p'_7, p_8, \dots, p_m)$.
  \begin{proof}

    For each $k$-gonal face of $P$ one can create a $(2, 1)$-expansion by adding pentagons and heptagons only. Put a ring of pentagons around each face of $P$ and around these pentagons another ring of heptagons, as seen in \autoref{fig:case57:img1}.\\

    \begin{tikzfigure}{\label{fig:case57:img1}}{Adding a ring of pentagons and a ring of heptagons around each face}
      \matrix (m) [ column sep=1cm] {
        \begin{scope}[xscale=1.0, yscale=0.866]
          \filldraw[fill=gray!50!white] (-0.75, -0.5) -- (-0.5, -0.5) -- (-0.25, 0) -- (-0.5, 0.5) -- (-0.75, 0.5);
          \filldraw[fill=gray!50!white] (0.75, 0.5) -- (0.5, 0.5) -- (0.25, 0) -- (0.5, -0.5) -- (0.75, -0.5);
          \filldraw[fill=gray!50!white] (-0.375, -0.75) -- (-0.5, -0.5) -- (-0.25, 0) -- (0.25, 0) -- (0.5, -0.5) -- (0.375, -0.75);
          \filldraw[fill=gray!50!white] (-0.375, 0.75) -- (-0.5, 0.5) -- (-0.25, 0) -- (0.25, 0) -- (0.5, 0.5) -- (0.375, 0.75);
        \end{scope}
        &
        \begin{scope}[xscale=1.0, yscale=0.866]
          \filldraw[fill=gray!50!white] (-0.375, -3.25) --  (-0.5, -3) -- (-0.25, -2.5) -- (0.25, -2.5) -- (0.5, -3) -- (0.375, -3.25);
          \filldraw[fill=gray!50!white] (-0.375, 3.25) --  (-0.5, 3) -- (-0.25, 2.5) -- (0.25, 2.5) -- (0.5, 3) -- (0.375, 3.25);
          \filldraw[fill=gray!50!white] (4.5, 0.5) -- (4.25, 0.5) -- (4, 0) -- (4.25, -0.5) -- (4.5, -0.5);
          \filldraw[fill=gray!50!white] (-4.5, 0.5) -- (-4.25, 0.5) -- (-4, 0) -- (-4.25, -0.5) -- (-4.5, -0.5);

          \draw (-1.5, 0) -- (-1, 0.5) -- (1, -0.5) -- (1.5, 0);
          \draw (-1.5, 0) -- (-1.375, -0.75) -- (-3.125, -2.25) -- (-3, -3);
          \draw[very thick] (1.5, 0) -- (1.375, 0.75) -- (3.125, 2.25) -- (3, 3);
          \draw[very thick] (-1.5, 0) -- (-2.125, 0.25) -- (-2.375, 2.75) -- (-3, 3);
          \draw (1.5, 0) -- (2.125, -0.25) -- (2.375, -2.75) -- (3, -3);
          \draw[very thick] (-3, -3) -- (-3.5, -3.5);
          \draw[very thick] (-3.5, -3.5) -- (-4, -3.25);
          \draw[very thick] (3, 3) -- (3.5, 3.5);
          \draw[very thick] (3.5, 3.5) -- (4, 3.25);
          \draw[very thick] (-3, 3) -- (-3.5, 2.5);
          \draw[very thick] (-3.5, 2.5) -- (-4, 2.75);
          \draw[very thick] (3, -3) -- (3.5, -2.5);
          \draw[very thick] (3.5, -2.5) -- (4, -2.75);
          \draw[very thick] (-3, -3) -- (-2.375, -3.25);
          \draw[very thick] (-2.375, -3.25) -- (-2.275, -4);
          \draw[very thick] (3, 3) -- (2.375, 3.25);
          \draw[very thick] (2.375, 3.25) -- (2.275, 4);
          \draw[very thick] (3, -3) -- (3.125, -3.75);
          \draw[very thick] (3.125, -3.75) -- (2.75, -4);
          \draw[very thick] (-3, 3) -- (-3.125, 3.75);
          \draw[very thick] (-3.125, 3.75) -- (-2.75, 4);

          \draw (-1, 0.5) -- (-1, 1)-- (-1.2, 2.2) -- (-2, 3);
          \draw (1, -0.5) -- (1, -1) -- (1.2, -2.2) -- (2, -3);
          \draw (-1.375, -0.75) -- (-1, -1) -- (0, -1.4) -- (1, -1);
          \draw (1.375, 0.75) -- (1, 1) -- (0, 1.4) -- (-1, 1);
          \draw (-3.125, -2.25) -- (-3.5, -2);
          \draw (3.125, 2.25) -- (3.5, 2);
          \draw (-2.125, 0.25) -- (-2.5, 0) -- (-3.3, -0.8) -- (-3.5, -2);
          \draw (2.125, -0.25) -- (2.5, 0) -- (3.3, 0.8) -- (3.5, 2);
          \draw (-2.375, 2.75) -- (-2, 3);
          \draw (2.375, -2.75) -- (2, -3);
          \draw (-3.5, 2.5) -- (-3.5, 2) -- (-3.3, 0.8) -- (-2.5, 0);
          \draw (3.5, -2.5) -- (3.5, -2) -- (3.3, -0.8) -- (2.5, 0);

          \draw (-3.5, -3.5) -- (-3.5, -4);
          \draw (3.5, 3.5) -- (3.5, 4);
          \draw (-2.375, -3.25) -- (-2, -3) -- (-1.2, -2.2) -- (-1, -1);
          \draw (2.375, 3.25) -- (2, 3) -- (1.2, 2.2) -- (1, 1);
          \draw (-3.125, 3.75) -- (-3.5, 4);
          \draw (3.125, -3.75) -- (3.5, -4);

          \draw (-0.5, -3) -- (-1.2, -2.2);
          \draw (0.5, 3) -- (1.2, 2.2);
          \draw (-0.25, -2.5) -- (0, -1.4);
          \draw (0.25, 2.5) -- (0, 1.4);
          \draw (0.25, -2.5) -- (1.2, -2.2);
          \draw (-0.25, 2.5) -- (-1.2, 2.2);
          \draw (0.5, -3) -- (1.2, -3.8) -- (2, -3);
          \draw (-0.5, 3) -- (-1.2, 3.8) -- (-2, 3);
          \draw (-2, -3) -- (-1.2, -3.8);
          \draw (2, 3) -- (1.2, 3.8);

          \draw (4.25, -0.5) -- (3.3, -0.8);
          \draw (-4.25, 0.5) -- (-3.3, 0.8);
          \draw (4, 0) -- (3.3, 0.8);
          \draw (-4, 0) -- (-3.3, -0.8);
          \draw (4.25, 0.5) -- (4.5, 1.6) -- (3.5, 2);
          \draw (-4.25, -0.5) -- (-4.5, -1.6) -- (-3.5, -2);
          \draw (3.5, -2) -- (4.5, -1.6);
          \draw (-3.5, 2) -- (-4.5, 1.6);

          \draw [very thick] (-2.275, -4) -- (-2.375, -3.25) -- (-3, -3) -- (-3.125, -2.25) -- (-1.375, -0.75) -- (-1.5, 0) -- (-1, 0.5) -- (1, -0.5) -- (1.5, 0) -- (2.125, -0.25) -- (2.375, -2.75) -- (3, -3) -- (3.125, -3.75);

        \end{scope}
        \\
      };
    \end{tikzfigure}

    Now fix a set $S$ of hexagons to replace, $|S| = p_6 - a$. For each of these hexagons one can replace the before mentioned ring structure circumscribing the hexagon by another arrangement as seen in \autoref{fig:case57:img3}.

    \begin{tikzfigure}{\label{fig:case57:img3}}{Replacement figure in the case of hexagons}
      \matrix (m) [ column sep=1cm] {
        \begin{scope}[xscale=1.0, yscale=0.866, scale=0.5]
          \draw[very thick] (-1.5, 0) -- (-1, -1) -- (0, -1) -- (0.5, -2) -- (1.5, -2) -- (2, -1) -- (3, -1) -- (3.5, 0) -- (3, 1) -- (3.5, 2) -- (3, 3) -- (2, 3) -- (1.5, 4) -- (0.5, 4) -- (0, 3) -- (-1, 3) -- (-1.5, 2) -- (-1, 1) -- cycle;
          \filldraw[fill=gray!50!white] (0.25, 0.5) -- (1, 0) -- (1.75, 0.5) -- (1.75, 1.5) -- (1, 2) -- (0.25, 1.5) -- cycle;
          \draw (0, -1) -- (0.25, -0.5) -- (1, -0.25) -- (1.75, -0.5) (1, -0.25) -- (1, 0);
          \draw (2, -1) -- (1.75, -0.5) -- (1.9375, 0.375) -- (2.5, 1) (1.9375, 0.375) -- (1.75, 0.5);
          \draw (3, 1) -- (2.5, 1) -- (1.9375, 1.625) -- (1.75, 2.5) (1.9375, 1.625) -- (1.75, 1.5);
          \draw (2, 3) -- (1.75, 2.5) -- (1, 2.25) -- (0.25, 2.5) (1, 2.25) -- (1, 2);
          \draw (0, 3) -- (0.25, 2.5) -- (0.0625, 1.625) -- (-0.5, 1) (0.0625, 1.625) -- (0.25, 1.5);
          \draw (-1, 1) -- (-0.5, 1) -- (0.0625, 0.375) -- (0.25, -0.5) (0.0625, 0.375) -- (0.25, 0.5);
        \end{scope};
        &
        \begin{scope}[xscale=1.0, yscale=0.866, scale=0.5]
          \draw (-1.5, 0) -- ++(0.5, -1) -- ++(1, 0) -- ++(0.25, 1.5) -- ++(-1.25, 0.5) -- ++(-0.5, -1);
          \draw (0, -1) -- ++(0.5, -1) -- ++(1, 0) -- ++(0.5, 1) -- ++(-0.25, 1.5) -- ++(-0.75, 0.5) -- ++(-0.75, -0.5);
          \draw (1.75, 0.5) -- ++(0.25, -1.5) -- ++(1, 0) -- ++(0.5, 1) -- ++(-0.5, 1) -- ++(-1.25, -0.5);
          \draw (1, 1) -- ++(0.75, -0.5) -- ++(1.25, 0.5) -- ++(0.5, 1) -- ++(-0.5, 1) -- ++(-1, 0) -- ++(-1, -1);
          \draw (0, 3) -- ++(1, -1) -- ++(1, 1) -- ++(-0.5, 1) -- ++(-1, 0) -- ++(-0.5, -1);
          \draw (-1.5, 2) -- ++(0.5, -1) -- ++(1.25, -0.5) -- ++(0.75, 0.5) -- ++(0, 1) -- ++(-1, 1) -- ++(-1, 0) -- ++(-0.5, -1);
          \draw[very thick] (-1.5, 0) -- (-1, -1) -- (0, -1) -- (0.5, -2) -- (1.5, -2) -- (2, -1) -- (3, -1) -- (3.5, 0) -- (3, 1) -- (3.5, 2) -- (3, 3) -- (2, 3) -- (1.5, 4) -- (0.5, 4) -- (0, 3) -- (-1, 3) -- (-1.5, 2) -- (-1, 1) -- cycle;
        \end{scope};

        \\
      };
    \end{tikzfigure}

    Thus every face of $P$ is replaced by some $(2, 1)$-expansion of itself. Since $(2, 1)$ is a self-fitting tuple, using \autoref{thm:construction:patch} results in a $3$-valent polyhedron $P'$. Its $p$-vector $p'$ differs by construction only in the number of pentagons, hexagons and heptagons from $p$. $a = p'_6$, since one replaced the required amount of hexagons in $P$, while $p'_5 - p_5 = p'_7 - p_7$ follows from \autoref{eq:valence:3}:
    \begin{align*}
      &0 = \sum_{k=3}^n \left(6 - k \right) p'_k - \sum_{k=3}^n \left(6 - k \right) p_k \\
      \implies & (p'_5 - p'_7) - (p_5 -p_7) = 0
    \end{align*}
  \end{proof}
\end{construction}

\begin{corollary}
  Let $p = (p_3, p_4, p_5, \dots, p_n)$ be a given sequence satisfying \autoref{eq:valence:3}, then $p$ is $[5, 7]$-$3$-realizable.
  \begin{proof}
    The proof starts by using the construction given by Eberhard's \autoref{thm:eberhard:3}. This results in a realization of $p$ which has many more hexagons then needed. The next step is to remove these hexagons and replace them by pentagons and heptagons. This is done by \autoref{thm:construction:5:7}. The resulting polyhedron has the required amount of polygons of each shape.\\
  \end{proof}
\end{corollary}
