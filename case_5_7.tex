\section{The case $[5, 7]_3$}
This theorem is already established in \cite{devos2010eberhard}. Here presented is a much simpler proof:


\begin{construction}
  Given a $3$-valent polyhedron $P$ with $p$-vector $(p_3, p_4, p_5, p_6, p_7, p_8, \dots, p_m)$ and $p'_6 \leq p_6$, there exists another $3$-valent polyhedron $P'$ with $p$-vector $(p_3, p_4, p'_5, p'_6, p'_7, p_8, \dots, p_m)$, where $p'_5 - p_5 = p'_7 - p_7$.
  \begin{proof}
    The construction consists of two steps. In the first step, each face of $P$ is separated by a ring of pentagons and heptagons. The second step is then used to replace the resulting structure around each hexagon to give an alternative construction only using pentagons and heptagons, eliminating the hexagon in the process.\\

    

    \begin{tikzfigure}{\label{fig:case57:img1}}
      \matrix (m) [ column sep=1cm] {

        \begin{scope}[scale=0.25] 
          \filldraw[fill=gray!50!white] (-3, 0) -- (0, -3) -- (3, 0) -- (0, 3) -- (-3, 0);
          %% \draw (-3, 0) -- (-6, 0);
          %% \draw (0, -3) -- (0, -6);
          %% \draw (3, 0) -- (6, 0);
          %% \draw (0, 3) -- (0, 6);
          %% \draw (-6, 0) -- (-7, -3) -- (-3, -7) -- (0, -6) -- (3, -7) -- (7, -3) -- (6, 0) -- (7, 3) -- (3, 7) -- (0, 6) -- (-3, 7) -- (-7, 3) -- (-6, 0);
        \end{scope}
        &
        \begin{scope}[scale=0.25] 
          \draw (-6, 0) -- (-7, -3) -- (-3, -7) -- (0, -6) -- (3, -7) -- (7, -3) -- (6, 0) -- (7, 3) -- (3, 7) -- (0, 6) -- (-3, 7) -- (-7, 3) -- (-6, 0);
          \draw (-5, 0) -- (-6, 0);
          \draw (0, -5) -- (0, -6);
          \draw (5, 0) -- (6, 0);
          \draw (0, 5) -- (0, 6);

          \draw (-5, 0) -- (-2, -2) -- (0, -5) -- (2, -2) -- (5, 0) -- (2, 2) -- (0, 5) -- (-2, 2) -- (-5, 0);
          \draw (-2, -2) -- (-1, -1);
          \draw (2, -2) -- (1, -1);
          \draw (2, 2) -- (1, 1);
          \draw (-2, 2) -- (-1, 1);
          
          \filldraw[fill=gray!50!white] (-1, -1) -- (1, -1) -- (1, 1) -- (-1, 1) -- (-1, -1);

        \end{scope}
        
        \\
      };
    \end{tikzfigure}
    
  \end{proof}
\end{construction}

\begin{theorem}
  Let $p = (p_3, p_4, p_5, \dots, p_n)$ be a given sequence satisfying \autoref{eq:valence:3}, then there exists $r \in \nats$ for which $p + r [5, 7]_3$ is $3$-realizable.
  \begin{proof}
    The proof starts by using the construction given by Eberhards \autoref{thm:eberhard:3}. This results in a realization of $p$ which has many more hexagons then needed. The next step is to remove these hexagons and replace them by pentagons and heptagons. To further proceed, one adds a ring of hexagons around each polygon in the realization, these will be replaced immediately but serve the purpose of adding space around each polygon which can be used for replacement with pentagons and heptagons. The process is detailed in \autoref{fig:case57:img1}.\\

    \begin{tikzfigure}{\label{fig:case57:img2}}
      \matrix (m) [ column sep=1cm] {
        \begin{scope}[xscale=1.0, yscale=0.866]
          \filldraw[fill=gray!50!white] (0, 1) -- ++(0.5, -1) -- ++(1, 0) -- ++(0.5, 1) -- ++(-0.5, 1) -- ++(-1, 0) -- ++(-0.5, -1);
          \filldraw[fill=gray!50!white] (1.5, 0) -- ++(0.5, -1) -- ++(1, 0) -- ++(0.5, 1) -- ++(-0.5, 1) -- ++(-1, 0) -- ++(-0.5, -1);
          \filldraw[fill=gray!50!white] (1.5, 2) -- ++(0.5, -1) -- ++(1, 0) -- ++(0.5, 1) -- ++(-0.5, 1) -- ++(-1, 0) -- ++(-0.5, -1);
          \draw[very thick] (1.5, 0) -- ++(0.5, 1) -- ++(-0.5, 1);
          \draw[very thick] (2, 1) -- ++(1, 0);
        \end{scope}
        &


        \begin{scope}[rotate=45, xscale=1.0, yscale=0.866, scale=0.5] 
          \filldraw[fill=gray!50!white] (0, 1) -- ++(0.5, -1) -- ++(1, 0) -- ++(0.5, 1) -- ++(-0.5, 1) -- ++(-1, 0) -- ++(-0.5, -1);
          \draw (-1.5, 0) -- ++(0.5, -1) -- ++(1, 0) -- ++(0.5, 1) -- ++(-0.5, 1) -- ++(-1, 0) -- ++(-0.5, -1);
          \draw (0, -1) -- ++(0.5, -1) -- ++(1, 0) -- ++(0.5, 1) -- ++(-0.5, 1) -- ++(-1, 0) -- ++(-0.5, -1);
          \draw (1.5, 0) -- ++(0.5, -1) -- ++(1, 0) -- ++(0.5, 1) -- ++(-0.5, 1) -- ++(-1, 0) -- ++(-0.5, -1);
          \draw (1.5, 2) -- ++(0.5, -1) -- ++(1, 0) -- ++(0.5, 1) -- ++(-0.5, 1) -- ++(-1, 0) -- ++(-0.5, -1);
          \draw (0, 3) -- ++(0.5, -1) -- ++(1, 0) -- ++(0.5, 1) -- ++(-0.5, 1) -- ++(-1, 0) -- ++(-0.5, -1);
          \draw (-1.5, 2) -- ++(0.5, -1) -- ++(1, 0) -- ++(0.5, 1) -- ++(-0.5, 1) -- ++(-1, 0) -- ++(-0.5, -1);

          \filldraw[fill=gray!50!white] (4.5, 0) -- ++(0.5, -1) -- ++(1, 0) -- ++(0.5, 1) -- ++(-0.5, 1) -- ++(-1, 0) -- ++(-0.5, -1);
          \draw (3, -1) -- ++(0.5, -1) -- ++(1, 0) -- ++(0.5, 1) -- ++(-0.5, 1) -- ++(-1, 0) -- ++(-0.5, -1);
          \draw (4.5, -2) -- ++(0.5, -1) -- ++(1, 0) -- ++(0.5, 1) -- ++(-0.5, 1) -- ++(-1, 0) -- ++(-0.5, -1);
          \draw (6, -1) -- ++(0.5, -1) -- ++(1, 0) -- ++(0.5, 1) -- ++(-0.5, 1) -- ++(-1, 0) -- ++(-0.5, -1);
          \draw (6, 1) -- ++(0.5, -1) -- ++(1, 0) -- ++(0.5, 1) -- ++(-0.5, 1) -- ++(-1, 0) -- ++(-0.5, -1);
          \draw (4.5, 2) -- ++(0.5, -1) -- ++(1, 0) -- ++(0.5, 1) -- ++(-0.5, 1) -- ++(-1, 0) -- ++(-0.5, -1);
          \draw (3, 1) -- ++(0.5, -1) -- ++(1, 0) -- ++(0.5, 1) -- ++(-0.5, 1) -- ++(-1, 0) -- ++(-0.5, -1);
          
          \filldraw[fill=gray!50!white] (1.5, -4) -- ++(0.5, -1) -- ++(1, 0) -- ++(0.5, 1) -- ++(-0.5, 1) -- ++(-1, 0) -- ++(-0.5, -1);
          \draw (0, -5) -- ++(0.5, -1) -- ++(1, 0) -- ++(0.5, 1) -- ++(-0.5, 1) -- ++(-1, 0) -- ++(-0.5, -1);
          \draw (1.5, -6) -- ++(0.5, -1) -- ++(1, 0) -- ++(0.5, 1) -- ++(-0.5, 1) -- ++(-1, 0) -- ++(-0.5, -1);
          \draw (3, -5) -- ++(0.5, -1) -- ++(1, 0) -- ++(0.5, 1) -- ++(-0.5, 1) -- ++(-1, 0) -- ++(-0.5, -1);
          \draw (3, -3) -- ++(0.5, -1) -- ++(1, 0) -- ++(0.5, 1) -- ++(-0.5, 1) -- ++(-1, 0) -- ++(-0.5, -1);
          \draw (1.5, -2) -- ++(0.5, -1) -- ++(1, 0) -- ++(0.5, 1) -- ++(-0.5, 1) -- ++(-1, 0) -- ++(-0.5, -1);
          \draw (0, -3) -- ++(0.5, -1) -- ++(1, 0) -- ++(0.5, 1) -- ++(-0.5, 1) -- ++(-1, 0) -- ++(-0.5, -1);

          \draw[very thick] (0.5, -2) -- ++(1, 0) -- ++(0.5, 1) -- ++(1, 0) -- ++(0.5, 1) -- ++(-0.5, 1) -- ++(0.5, 1);
          \draw[very thick] (3, -1) -- ++(0.5, -1) -- ++(1, 0) -- ++(0.5, -1);
        \end{scope};
        \\
      };
    \end{tikzfigure}
    To see, that each ring actually line up correctly with its neighbors it could come in handy to review the replacement of an edge, see \autoref{fig:case57:img2}. Each line is substitued by a zigzag line of length three. Since the sphere is orientable one can choose a fitting orientation of this zigzag, say first to the left when viewed from each vertex.

    \begin{tikzfigure}
      

    \end{tikzfigure}

    One now chooses a set $S$ of hexagons one wants to keep, $|S| = p_6$. Each hexagon not in $S$ and the surrounding six hexagons are replaced by the following structure: 

    \begin{tikzfigure}{\label{fig:case57:img3}}
      \matrix (m) [ column sep=1cm] {
        \begin{scope}[xscale=1.0, yscale=0.866, scale=0.5]
          \filldraw[fill=gray!50!white] (0, 1) -- ++(0.5, -1) -- ++(1, 0) -- ++(0.5, 1) -- ++(-0.5, 1) -- ++(-1, 0) -- ++(-0.5, -1);
          \draw (-1.5, 0) -- ++(0.5, -1) -- ++(1, 0) -- ++(0.5, 1) -- ++(-0.5, 1) -- ++(-1, 0) -- ++(-0.5, -1);
          \draw (0, -1) -- ++(0.5, -1) -- ++(1, 0) -- ++(0.5, 1) -- ++(-0.5, 1) -- ++(-1, 0) -- ++(-0.5, -1);
          \draw (1.5, 0) -- ++(0.5, -1) -- ++(1, 0) -- ++(0.5, 1) -- ++(-0.5, 1) -- ++(-1, 0) -- ++(-0.5, -1);
          \draw (1.5, 2) -- ++(0.5, -1) -- ++(1, 0) -- ++(0.5, 1) -- ++(-0.5, 1) -- ++(-1, 0) -- ++(-0.5, -1);
          \draw (0, 3) -- ++(0.5, -1) -- ++(1, 0) -- ++(0.5, 1) -- ++(-0.5, 1) -- ++(-1, 0) -- ++(-0.5, -1);
          \draw (-1.5, 2) -- ++(0.5, -1) -- ++(1, 0) -- ++(0.5, 1) -- ++(-0.5, 1) -- ++(-1, 0) -- ++(-0.5, -1);
        \end{scope};

        &

        \begin{scope}[xscale=1.0, yscale=0.866, scale=0.5] 
          \draw (-1.5, 0) -- ++(0.5, -1) -- ++(1, 0) -- ++(0.25, 1.5) -- ++(-1.25, 0.5) -- ++(-0.5, -1);
          \draw (0, -1) -- ++(0.5, -1) -- ++(1, 0) -- ++(0.5, 1) -- ++(-0.25, 1.5) -- ++(-0.75, 0.5) -- ++(-0.75, -0.5);
          \draw (1.75, 0.5) -- ++(0.25, -1.5) -- ++(1, 0) -- ++(0.5, 1) -- ++(-0.5, 1) -- ++(-1.25, -0.5);
          \draw (1, 1) -- ++(0.75, -0.5) -- ++(1.25, 0.5) -- ++(0.5, 1) -- ++(-0.5, 1) -- ++(-1, 0) -- ++(-1, -1);
          \draw (0, 3) -- ++(1, -1) -- ++(1, 1) -- ++(-0.5, 1) -- ++(-1, 0) -- ++(-0.5, -1);
          \draw (-1.5, 2) -- ++(0.5, -1) -- ++(1.25, -0.5) -- ++(0.75, 0.5) -- ++(0, 1) -- ++(-1, 1) -- ++(-1, 0) -- ++(-0.5, -1);
        \end{scope};
        
        \\
      };
    \end{tikzfigure}

    For the remaining hexagons and all other kinds of polygons this ring structure is replaced differently, for each $n$-gon $n$ pentagons and $n$ heptagons replace the ring of hexagons:


    This finishes the proof.
  \end{proof}
\end{theorem}


