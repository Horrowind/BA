\usepackage[utf8]{inputenc}
%\usepackage[ngerman]{babel}
\usepackage[english]{babel}
\usepackage{hyperref}
\usepackage{amsmath}
\usepackage{amsthm}
\usepackage{amssymb}
\usepackage{mathrsfs}
\usepackage{amstext}
\usepackage{amsfonts}
\usepackage{mathtools}
\usepackage{enumitem}
\usepackage{chngcntr}
\usepackage{caption}
\counterwithin{figure}{section}

%\usepackage{extarrows}

\usepackage{tikz}
\usetikzlibrary{%
    shapes.geometric,
    matrix,
    intersections
}
\usepackage{shapepar}

\setlength{\parindent}{0pt}
\renewcommand{\baselinestretch}{1.0}
\usepackage[inner=2cm,outer=2cm,top=1.5cm,bottom=1.5cm,includeheadfoot]{geometry}
\theoremstyle{definition}

% figure enviroments
\captionsetup[FLOAT_TYPE]{labelformat=simple, labelsep=none}
\newenvironment{tikzfigure}[1]
    {\begin{figure}[h!] \caption{} #1 \centering  \begin{tikzpicture}}
    {\end{tikzpicture} \end{figure}}





% math environments
\newtheorem{theorem}{Theorem}[section]
\newtheorem*{theorem*}{Theorem}
\newtheorem{lemma}[theorem]{Lemma}
\newtheorem*{lemma*}{Lemma}
\newtheorem{problem}[theorem]{Problem}
\newtheorem*{problem*}{Problem}
\newtheorem{notation}[theorem]{Notation}
\newtheorem*{notation*}{Notation}
\newtheorem{example}[theorem]{Example}
\newtheorem*{example*}{Example}
\newtheorem{remark}[theorem]{Remark}
\newtheorem*{remark*}{Remark}
\newtheorem{conjecture}[theorem]{Conjecture}
\newtheorem*{conjecture*}{Conjecture}
\newtheorem{definition}[theorem]{Definition}
\newtheorem*{definition*}{Definition}
\newtheorem{corollary}[theorem]{Corollary}
\newtheorem*{corollary*}{Corollary}
\newtheorem{construction}[theorem]{Construction}
\newtheorem*{construction*}{Construction}

\renewcommand*{\theenumi}{\thetheorem\,(\roman{enumi})}%
\renewcommand*{\labelenumi}{\thetheorem\,(\roman{enumi})}%

% Operatoren

\newcommand{\piece}[1]{\textcircled{\tiny{#1}}}
%\newcommand{\deg}{\operatorname{deg}}
% Abkürzungen
\newcommand{\nats}{\mathbb{N}}
\newcommand{\wholes}{\mathbb{Z}}

\newcommand{\eps}{\varepsilon}
\renewcommand{\phi}{\varphi}

% Things for equations  
\numberwithin{equation}{section}
\setcounter{subsection}{0}

%\renewcommand{\labelenumi}{(\alph{enumi})}
