\section{The $5$-valent case $[2 \times 3, 4]$}
For some stronger result in the field of $5$-valent polyhedra see \cite{trenkler1975face}. Here established are only some low-hanging fruits based on the mechanism seen before. The proof utilizes Eberhard's \autoref{thm:eberhard:4} for construction of the new solid.
\begin{theorem}
  Let $p = (p_3, p_4, p_5, \dots, p_n)$ be a given sequence satisfying \autoref{eq:valence:5}, then $p$ is $[2\times3, 4]$-$5$-realizable.
  \begin{proof}
    Beginning with equation \autoref{eq:valence:5} one retrieves the following:
    \begin{align*}
      &\sum_{k=3}^n \left( \frac{10}{3} - k \right) p_k = \frac{20}{3} \\
      \implies & \frac{p_3}{3} = \frac{20}{3} + \sum_{k=4}^n \left(k - \frac{10}{3} \right) p_k \geq \frac{20}{3} + \frac{2}{3} \sum_{k=4}^n p_k \\
      \implies & p_3 \geq \frac{20}{3} + \frac{2}{3} \sum_{k=3}^n p_k \geq - \frac{4}{3} + \frac{2}{3} \sum_{k=3}^n p_k
    \end{align*}
    Therefore one can assume, that $p'_3 := p_3 + \frac{4}{3} - \frac{2}{3} \sum_{k=3}^n p_k \geq 0$ and for $p'_k = p_k$, $(k\geq 4)$ the resulting sequence satisfies:
    \begin{align*}
      \sum_{k=3}^n (4 - k) p'_k =&~ p_3 + \frac{4}{3} - \frac{2}{3} \sum_{k=3}^n p_k + \sum_{k=4}^n (4 - k) p_k \\
      = \frac{4}{3} + \frac{1}{3} p_3 + \sum_{k=4}^n \left(k - \frac{10}{3} \right) p_k =&~ \frac{4}{3} + \sum_{k=3}^n \left(k - \frac{10}{3} \right) p_k = \frac{4}{3} + \frac{20}{3} = 8.
    \end{align*}
    Hence the sequence $p'$ is $[4]$-$4$-realizable by Eberhard's Theorem \autoref{thm:eberhard:4}. Let $P'$ be this realization with $e'$ edges and $v'$ vertices. One can now use $P'$ to create a realization with valence $5$ by inserting further triangles and quadrangles. This is done by replacing each edge of the realization by two triangles and each vertex by one square as shown in \autoref{fig:case34:img1}. For this fix an orientation on $P'$ and replace each edge by two triangles so that around each vertex the same mill-shaped figure (shaded in dark gray in \autoref{fig:case34:img1}) emerge. By fixing the orientation it is prevented that mirrored images of the mill-shape appear, thus the construction is defined.

    \begin{tikzfigure}{\label{fig:case34:img1}}{Building a $4$-valent polyhedron out of a $3$-valent one by adding triangles and quadrangles}
      \matrix (m) [ column sep=1cm] {
        \begin{scope}[scale=0.5]

          \filldraw[fill=gray!50!white] (-3.5, -0.5) -- (-3, 0) -- (-1, 0) -- (-1, -2) -- (-1.5, -2.5);
          \filldraw[fill=gray!50!white] (3.5, -0.5) -- (3, 0) -- (1, 0) -- (1, -2) -- (1.5, -2.5);
          \filldraw[fill=gray!50!white] (3.5, 0.5) -- (3, 0) -- (1, 0) -- (1, 2) -- (1.5, 2.5);
          \filldraw[fill=gray!50!white] (-3.5, 0.5) -- (-3, 0) -- (-1, 0) -- (-1, 2) -- (-1.5, 2.5);
          \filldraw[fill=gray!50!white] (-0.5, -2.5) -- (-1, -2) -- (-1, 0) -- (1, 0) -- (1, -2) -- (0.5, -2.5);
          \filldraw[fill=gray!50!white] (-0.5, 2.5) -- (-1, 2) -- (-1, 0) -- (1, 0) -- (1, 2) -- (0.5, 2.5);

          \draw (-3.5, 0) -- (-3, 0);
          \draw (3.5, 0) -- (3, 0);
          \draw (-1, 2.5) -- (-1, 2);
          \draw (-1, -2.5) -- (-1, -2);
          \draw (1, 2.5) -- (1, 2);
          \draw (1, -2.5) -- (1, -2);

          \draw[very thick] (-3, 0) -- (3, 0);
          \draw[very thick] (-1, -2) -- (-1, 2);
          \draw[very thick] (1, -2) -- (1, 2);

        \end{scope}
        &
        \begin{scope}[scale=0.5]
          \filldraw[fill=gray!50!white] (-5.5, -1.5) -- (-5, -1) -- (-3, -1) -- (-3, -3) -- (-3.5, -3.5);
          \filldraw[fill=gray!50!white] (5.5, -1.5) -- (5, -1) -- (3, -1) -- (3, -3) -- (3.5, -3.5);
          \filldraw[fill=gray!50!white] (5.5, 1.5) -- (5, 1) -- (3, 1) -- (3, 3) -- (3.5, 3.5);
          \filldraw[fill=gray!50!white] (-5.5, 1.5) -- (-5, 1) -- (-3, 1) -- (-3, 3) -- (-3.5, 3.5);
          \filldraw[fill=gray!50!white] (-0.5, -3.5) -- (-1, -3) -- (-1, -1) -- (1, -1) -- (1, -3) -- (0.5, -3.5);
          \filldraw[fill=gray!50!white] (-0.5, 3.5) -- (-1, 3) -- (-1, 1) -- (1, 1) -- (1, 3) -- (0.5, 3.5);

          \filldraw[fill=gray!75!white] (-5, -1) -- (-3, 1) -- (-3, 3) -- (-1, 1) -- (1, 1) -- (-1, -1) -- (-1, -3) -- (-3, -1) -- cycle;

          \draw[very thick] (-5, -1) -- (5, -1);
          \draw[very thick] (-5, 1) -- (5, 1);
          \draw[very thick] (-3, -3) -- (-3, 3);
          \draw[very thick] (-1, -3) -- (-1, 3);
          \draw[very thick] (1, -3) -- (1, 3);
          \draw[very thick] (3, -3) -- (3, 3);
          \draw[very thick] (-5, -1) -- (-3, 1);
          \draw[very thick] (-1, -1) -- (1, 1);
          \draw[very thick] (3, -1) -- (5, 1);
          \draw[very thick] (-3, -1) -- (-1, -3);
          \draw[very thick] (-3, 3) -- (-1, 1);
          \draw[very thick] (1, -1) -- (3, -3);
          \draw[very thick] (1, 3) -- (3, 1);

          \draw[very thick] (-5, -1) -- (-5, 1); \draw[very thick] (5, -1) -- (5, 1);
          \draw[very thick] (-3, -3) -- (-1, -3); \draw[very thick] (1, -3) -- (3, -3);
          \draw[very thick] (-3, 3) -- (-1, 3); \draw[very thick] (1, 3) -- (3, 3);

          \draw (-3, -3.5) -- (-3, -3) -- (-2.5, -3.5);
          \draw (-1, -3.5) -- (-1, -3);
          \draw (1, -3.5) -- (1, -3) -- (1.5, -3.5);
          \draw (3, -3.5) -- (3, -3);
          \draw (-3, 3.5) -- (-3, 3);
          \draw (-1, 3.5) -- (-1, 3) -- (-1.5, 3.5);
          \draw (1, 3.5) -- (1, 3);
          \draw (3, 3.5) -- (3, 3) -- (2.5, 3.5);

          \draw (-5.5, -1) -- (-5, -1);
          \draw (-5.5, 1) -- (-5, 1) -- (-5.5, 0.5);
          \draw (5.5, -1) -- (5, -1) -- (5.5, -0.5);
          \draw (5.5, 1) -- (5, 1);

        \end{scope}
        \\
        };
    \end{tikzfigure}
    The resulting polyhedron $P$ is $5$-valent, let $p''$ be its $p$-vector. It differs from $p$ only by the number of triangles and quadrangles. Since equation \autoref{eq:valence:5} holds for $p$ as well as for $p''$, this gives
    \begin{align*}
      & \sum_{k=3}^n \left( \frac{10}{3} - k \right) p_k'' - \sum_{k=3}^n \left( \frac{10}{3} - k \right) p_k = 0 \\
      \implies& \frac{1}{3} (p''_3 - p_3) = \frac{2}{3} (p''_4 - p_4),
    \end{align*}
    hence $p + r [2 \times 3, 4] = p''$ is $5$-realizable for some $r \in \nats$.
  \end{proof}
\end{theorem}
