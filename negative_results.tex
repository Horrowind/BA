\section{Negative results}

\begin{theorem}\label{thm:nonexistence} A polygon is called a multi-$k$-gon provided its number of sides is divisible by $k$. Let $r \in \set{3, 4, 5}$, $k \in \set{2, 3, 4, 5}$. There exists no planar graph without monogons and digons with valence $r$ at each vertex where each but one face is a multi-$k$-gon. This special face will be called the exceptional face.

\begin{remark}
  When considering connected planar graphs some oddities can occur. A graph could for example have a face which visits a single vertex multiple times or two sides which in the end denote the same edge. One of these graphs can be seen in \autoref{fig:case3:6:example}.
  \begin{tikzfigure}{\label{fig:case3:6:example}}{}
    \draw (-1, 0) -- (-1, -1) -- (-1, 1) -- (1, 1) -- (1, -1) -- (-1, -1) -- (-1, 0) -- (0, 0) -- (0.5, -0.5) -- (0.5, 0.5) -- (0, 0);
  \end{tikzfigure}
  For the following considerations, one will count each of these occurrences separately, thus the example consists of a triangle, a decagon and a pentagon (as the outer face). It has eight vertices and nine edges.  
\end{remark}

\begin{proof}
The case $r=3$ is proven in \cite{ConvexPolytopes} (Theorem 13.4.2). If $r=4$ then \autoref{eq:valence:4} states that a minimum of $8$ triangles has to occur, therefore $k=3$. Let $G$ be a planar graph without monogons and digons where each but one face is a multi-$3$-gon. The proof tries to reduce $G$ to a smaller graph with the same properties, leading to an infinite descent or to a counterexample with exactly two exceptional faces (which are not multi-$k$-gons). The first thing to notice is that one can assume the number of edges of the exceptional face to be four or five. In fact, by changing two edges it is possible to ``cut out'' a square or a pentagon from any larger exceptional faces.

  In \autoref{fig:case3:6:img4} let $B$ be the exceptional face with $b$ edges. Under the transformation $B$ is reduced to a square or a pentagon, depending on whether $b \equiv 1 (\operatorname{mod} 3)$ or $b \equiv 2 (\operatorname{mod} 3)$. In the first case $E$ has $b - 4$ edges, in the second $b-5$. In either case $E$ is not an exceptional face, its number of edges is divisible by $3$. If $A$ and $C$ are different faces, the resulting graph is connected. As $A$ and $C$ have a number of edges which is divisible by $3$, so has the new face $D$. It remains the case that $A$ is the same face as $C$. Then the free space labeled with $D$ is separated by two cycles of length $d$ (denoting the length of ``left'' cycle) and $d'$ (denoting the length of ``right'' cycle). If $d$ is divisible by $3$, then the left component has exactly one exceptional face and has fewer edges than the original one. If $d$ is not divisible by $3$, then neither is $d'$, as their sum matches the sum of the number of edges of $A$ and $C$, thus the right component has exactly one exceptional face (the cycle labeled with $d'$) and fewer faces than $G$.
  \begin{tikzfigure}{\label{fig:case3:6:img4}}{}
    \matrix (m) [ column sep=1cm] {
      \begin{scope}
        \draw[loosely dotted] (4, 1) -- (3, 0.25);
        \draw (3, 0.25) -- (2, 0) -- (1, 0) -- (0, 1) -- (0, 2) -- (1, 3) -- (2, 3) -- (3, 2.75);
        \draw[loosely dotted] (3, 2.75) -- (4, 2);
        \draw (2, 0) -- ++(-0.2, -0.2)  ++(0.2, 0.2) -- ++(0.2, -0.2);
        \draw (1, 0) -- ++(-0.2, -0.2)  ++(0.2, 0.2) -- ++(0.2, -0.2);
        \draw (0, 2) -- ++(-0.2, -0.2)  ++(0.2, 0.2) -- ++(-0.2, 0.2);
        \draw (0, 1) -- ++(-0.2, -0.2)  ++(0.2, 0.2) -- ++(-0.2, 0.2);
        \draw (1, 3) -- ++(0.2, 0.2)  ++(-0.2, -0.2) -- ++(-0.2, 0.2);
        \draw (2, 3) -- ++(0.2, 0.2)  ++(-0.2, -0.2) -- ++(-0.2, 0.2);
        \node [above] at (1.5, 3) {A};
        \node at (1.5, 1.5) {B};
        \node [below] at (1.5, 0) {C};
      \end{scope}
      &
      \begin{scope}
        \draw[loosely dotted] (4, 1) -- (3, 0.25);
        \draw (3, 0.25) -- (2, 0) -- (2, 3) node [midway, above=10pt, left=-3pt] {d} -- (3, 2.75);
        \draw (1, 0) -- (0, 1) -- (0, 2) -- (1, 3) -- (1, 0) node [midway, above=10pt, right=-3pt] {d'};
        \draw[loosely dotted] (3, 2.75) -- (4, 2);
        \draw (2, 0) -- ++(-0.2, -0.2)  ++(0.2, 0.2) -- ++(0.2, -0.2);
        \draw (1, 0) -- ++(-0.2, -0.2)  ++(0.2, 0.2) -- ++(0.2, -0.2);
        \draw (0, 2) -- ++(-0.2, -0.2)  ++(0.2, 0.2) -- ++(-0.2, 0.2);
        \draw (0, 1) -- ++(-0.2, -0.2)  ++(0.2, 0.2) -- ++(-0.2, 0.2);
        \draw (1, 3) -- ++(0.2, 0.2)  ++(-0.2, -0.2) -- ++(-0.2, 0.2);
        \draw (2, 3) -- ++(0.2, 0.2)  ++(-0.2, -0.2) -- ++(-0.2, 0.2);
        \node at (1.5, 1.5) {D};
        \node at (3, 1.5) {E};
      \end{scope}
      \\
      \begin{scope}
        \draw[loosely dotted] (4, 1) -- (3, 0.25);
        \draw (3, 0.25) -- (2, 0) -- (1, 0) -- (0, 0.75) -- (-0.25, 1.5) -- (0, 2.25) -- (1, 3) -- (2, 3) -- (3, 2.75);
        \draw[loosely dotted] (3, 2.75) -- (4, 2);
        \draw (2, 0) -- ++(-0.2, -0.2)  ++(0.2, 0.2) -- ++(0.2, -0.2);
        \draw (1, 0) -- ++(-0.2, -0.2)  ++(0.2, 0.2) -- ++(0.2, -0.2);
        \draw (0, 2.25) -- ++(-0.2, -0.2)  ++(0.2, 0.2) -- ++(-0.2, 0.2);
        \draw (-0.25, 1.5) -- ++(-0.2, -0.2)  ++(0.2, 0.2) -- ++(-0.2, 0.2);
        \draw (0, 0.75) -- ++(-0.2, -0.2)  ++(0.2, 0.2) -- ++(-0.2, 0.2);
        \draw (1, 3) -- ++(0.2, 0.2)  ++(-0.2, -0.2) -- ++(-0.2, 0.2);
        \draw (2, 3) -- ++(0.2, 0.2)  ++(-0.2, -0.2) -- ++(-0.2, 0.2);
        \node [above] at (1.5, 3) {A};
        \node at (1.5, 1.5) {B};
        \node [below] at (1.5, 0) {C};
      \end{scope}
      &
      \begin{scope}
        \draw[loosely dotted] (4, 1) -- (3, 0.25);
        \draw (3, 0.25) -- (2, 0) -- (2, 3) node [midway, above=10pt, left=-3pt] {d} -- (3, 2.75);
        \draw (1, 0) -- (0, 0.75) -- (-0.25, 1.5) -- (0, 2.25) -- (1, 3) -- (1, 0) node [midway, above=10pt, right=-3pt] {d'};
        \draw[loosely dotted] (3, 2.75) -- (4, 2);
        \draw (2, 0) -- ++(-0.2, -0.2)  ++(0.2, 0.2) -- ++(0.2, -0.2);
        -- (-3, -3) -- (-3.125, -2.25)            \draw (1, 0) -- ++(-0.2, -0.2)  ++(0.2, 0.2) -- ++(0.2, -0.2);
        \draw (0, 2.25) -- ++(-0.2, -0.2)  ++(0.2, 0.2) -- ++(-0.2, 0.2);
        \draw (-0.25, 1.5) -- ++(-0.2, -0.2)  ++(0.2, 0.2) -- ++(-0.2, 0.2);
        \draw (0, 0.75) -- ++(-0.2, -0.2)  ++(0.2, 0.2) -- ++(-0.2, 0.2);
        \draw (1, 3) -- ++(0.2, 0.2)  ++(-0.2, -0.2) -- ++(-0.2, 0.2);
        \draw (2, 3) -- ++(0.2, 0.2)  ++(-0.2, -0.2) -- ++(-0.2, 0.2);
        \node at (1.5, 1.5) {D};
        \node at (3, 1.5) {E};
      \end{scope}
      \\
    };
  \end{tikzfigure}

  A similar construction (see \autoref{fig:case3:6:img5}) can be used to ``cut'' out a triangle from a larger non exceptional face ($B$). 
  
  \begin{tikzfigure}{\label{fig:case3:6:img5}}{}
    \matrix (m) [ column sep=1cm] {
      \begin{scope}
        \draw[loosely dotted] (4, 1) -- (3, 0.25);
        \draw (3, 0.25) -- (2, 0) -- (1, 0) -- (0, 1.5) -- (1, 3) -- (2, 3) -- (3, 2.75);
        \draw[loosely dotted] (3, 2.75) -- (4, 2);
        \draw (2, 0) -- ++(-0.2, -0.2)  ++(0.2, 0.2) -- ++(0.2, -0.2);
        \draw (1, 0) -- ++(-0.2, -0.2)  ++(0.2, 0.2) -- ++(0.2, -0.2);
        \draw (0, 1.5) -- ++(-0.2, -0.2)  ++(0.2, 0.2) -- ++(-0.2, 0.2);
        \draw (1, 3) -- ++(0.2, 0.2)  ++(-0.2, -0.2) -- ++(-0.2, 0.2);
        \draw (2, 3) -- ++(0.2, 0.2)  ++(-0.2, -0.2) -- ++(-0.2, 0.2);
        \node [above] at (1.5, 3) {A};
        \node at (1.5, 1.5) {B};
        \node [below] at (1.5, 0) {C};
      \end{scope}
      &
      \begin{scope}
        \draw[loosely dotted] (4, 1) -- (3, 0.25);
        \draw (3, 0.25) -- (2, 0) -- (2, 3) node [midway, above=10pt, left=-3pt] {d} -- (3, 2.75);
        \draw (1, 0) -- (0, 1.5) -- (1, 3) -- (1, 0) node [midway, above=10pt, right=-3pt] {d'};
        \draw[loosely dotted] (3, 2.75) -- (4, 2);
        \draw (2, 0) -- ++(-0.2, -0.2)  ++(0.2, 0.2) -- ++(0.2, -0.2);
        \draw (1, 0) -- ++(-0.2, -0.2)  ++(0.2, 0.2) -- ++(0.2, -0.2);
        \draw (0, 1.5) -- ++(-0.2, -0.2)  ++(0.2, 0.2) -- ++(-0.2, 0.2);
        \draw (1, 3) -- ++(0.2, 0.2)  ++(-0.2, -0.2) -- ++(-0.2, 0.2);
        \draw (2, 3) -- ++(0.2, 0.2)  ++(-0.2, -0.2) -- ++(-0.2, 0.2);
        \node at (1.5, 1.5) {D};
        \node at (3, 1.5) {E};
      \end{scope}
      \\
    };
  \end{tikzfigure}
  The argument is mostly the same as in the previous case. $E$ has three edges less than $B$, the number of whose is divisible by $3$. There is also a differentiation to be made whether $A$ and $C$ denote the same face. If not, then the resulting graph is connected and has a triangle. If they are the same, the graph separates and one of the components has exactly one exceptional face and has fewer edges than $G$.

  This step will be used for the up to eight or ten faces which share at least one vertex with the exceptional face (eight if it is a square, ten if it is a pentagon). One can therefore assume the exceptional face to be surrounded by triangles, else one could try the above step to ``cut out'' a triangle, resulting in fewer edges or in a triangle at the desired position. But the resulting graph is the same as the skeleton of an antiprism (see \autoref{fig:case3:6:img6} for the two antiprisms with a quadrangle base and a pentagonal base), which has one additional exceptional face. The steps taken did not introduce new exceptional faces, a contradiction.
  \begin{tikzfigure}{\label{fig:case3:6:img6}}{}
    \matrix (m) [ column sep=1cm] {
      \begin{scope}[scale=0.75]
        \draw (-1, -1) -- (-1, 1) -- (1, 1) -- (1, -1) -- (-1, -1) -- (0, -3) -- (1, -1) -- (3, 0) -- (1, 1) -- (0, 3) -- (-1, 1) -- (-3 , 0) -- (-1, -1) (0, -3) -- (3, 0) -- (0, 3) -- (-3, 0) -- (0, -3);
      \end{scope}
      &
      \begin{scope}[scale=0.75]
        \draw (0 : 2) -- (72 : 2) -- (144 : 2) -- (216 : 2) -- (288 : 2) -- (0 : 2) -- (36 : 3) -- (72 : 2) -- (108 : 3) -- (144 : 2) -- (180 : 3) -- (216 : 2) -- (252 : 3) -- (288 : 2) -- (324 : 3) -- (0 : 2) -- (36 : 3) -- (108 : 3) -- (180 : 3) -- (252 : 3) -- (324 : 3) -- (36 : 3);
      \end{scope}
      \\
    };
  \end{tikzfigure}

Now consider $r=5$. The proof is similar to the previous case. One can use the same two constructions from above with small alterations (the resulting pictures now have vertices of valence $5$ and not $4$) as seen in \autoref{fig:valence5:img4} and \autoref{fig:valence5:img5}. The same arguments as before apply when cutting a larger exceptional face to a smaller one. Thus, assume that the exceptional face has four or five sides.
    \begin{tikzfigure}{\label{fig:valence5:img4}}{}
      \matrix (m) [ column sep=1cm] {
        \begin{scope}
          \draw[loosely dotted] (4, 1) -- (3, 0.25);
          \draw (3, 0.25) -- (2, 0) -- (1, 0) -- (0, 1) -- (0, 2) -- (1, 3) -- (2, 3) -- (3, 2.75);
          \draw[loosely dotted] (3, 2.75) -- (4, 2);
          \draw (2, 0) -- ++(-0.2, -0.2)  ++(0.2, 0.2) -- ++(0.2, -0.2) ++(-0.2, 0.2) -- ++(0, -0.2);
          \draw (1, 0) -- ++(-0.2, -0.2)  ++(0.2, 0.2) -- ++(0.2, -0.2) ++(-0.2, 0.2) -- ++(0, -0.2);
          \draw (0, 2) -- ++(-0.2, -0.2)  ++(0.2, 0.2) -- ++(-0.2, 0.2) ++(0.2, -0.2) -- ++(-0.2, 0);
          \draw (0, 1) -- ++(-0.2, -0.2)  ++(0.2, 0.2) -- ++(-0.2, 0.2) ++(0.2, -0.2) -- ++(-0.2, 0);
          \draw (1, 3) -- ++(0.2, 0.2)  ++(-0.2, -0.2) -- ++(-0.2, 0.2) ++(0.2, -0.2) -- ++(0, 0.2);
          \draw (2, 3) -- ++(0.2, 0.2)  ++(-0.2, -0.2) -- ++(-0.2, 0.2) ++(0.2, -0.2) -- ++(0, 0.2);
          \node [above] at (1.5, 3) {A};
          \node at (1.5, 1.5) {B};
          \node [below] at (1.5, 0) {C};
        \end{scope}
        &
        \begin{scope}
          \draw[loosely dotted] (4, 1) -- (3, 0.25);
          \draw (3, 0.25) -- (2, 0) -- (2, 3) node [midway, above=10pt, left=-3pt] {d} -- (3, 2.75);
          \draw (1, 0) -- (0, 1) -- (0, 2) -- (1, 3) -- (1, 0) node [midway, above=10pt, right=-3pt] {d'};
          \draw[loosely dotted] (3, 2.75) -- (4, 2);
          \draw (2, 0) -- ++(-0.2, -0.2)  ++(0.2, 0.2) -- ++(0.2, -0.2) ++(-0.2, 0.2) -- ++(0, -0.2);
          \draw (1, 0) -- ++(-0.2, -0.2)  ++(0.2, 0.2) -- ++(0.2, -0.2) ++(-0.2, 0.2) -- ++(0, -0.2);
          \draw (0, 2) -- ++(-0.2, -0.2)  ++(0.2, 0.2) -- ++(-0.2, 0.2) ++(0.2, -0.2) -- ++(-0.2, 0);
          \draw (0, 1) -- ++(-0.2, -0.2)  ++(0.2, 0.2) -- ++(-0.2, 0.2) ++(0.2, -0.2) -- ++(-0.2, 0);
          \draw (1, 3) -- ++(0.2, 0.2)  ++(-0.2, -0.2) -- ++(-0.2, 0.2) ++(0.2, -0.2) -- ++(0, 0.2);
          \draw (2, 3) -- ++(0.2, 0.2)  ++(-0.2, -0.2) -- ++(-0.2, 0.2) ++(0.2, -0.2) -- ++(0, 0.2);
          \node at (1.5, 1.5) {D};
          \node at (3, 1.5) {E};
        \end{scope}
        \\
        \begin{scope}
          \draw[loosely dotted] (4, 1) -- (3, 0.25);
          \draw (3, 0.25) -- (2, 0) -- (1, 0) -- (0, 0.75) -- (-0.25, 1.5) -- (0, 2.25) -- (1, 3) -- (2, 3) -- (3, 2.75);
          \draw[loosely dotted] (3, 2.75) -- (4, 2);
          \draw (2, 0) -- ++(-0.2, -0.2)  ++(0.2, 0.2) -- ++(0.2, -0.2) ++(-0.2, 0.2) -- ++(0, -0.2);
          \draw (1, 0) -- ++(-0.2, -0.2)  ++(0.2, 0.2) -- ++(0.2, -0.2) ++(-0.2, 0.2) -- ++(0, -0.2);
          \draw (0, 2.25) -- ++(-0.2, -0.2)  ++(0.2, 0.2) -- ++(-0.2, 0.2) ++(0.2, -0.2) -- ++(-0.2, 0);
          \draw (-0.25, 1.5) -- ++(-0.2, -0.2)  ++(0.2, 0.2) -- ++(-0.2, 0.2) ++(0.2, -0.2) -- ++(-0.2, 0);
          \draw (0, 0.75) -- ++(-0.2, -0.2)  ++(0.2, 0.2) -- ++(-0.2, 0.2) ++(0.2, -0.2) -- ++(-0.2, 0);
          \draw (1, 3) -- ++(0.2, 0.2)  ++(-0.2, -0.2) -- ++(-0.2, 0.2) ++(0.2, -0.2) -- ++(0, 0.2);
          \draw (2, 3) -- ++(0.2, 0.2)  ++(-0.2, -0.2) -- ++(-0.2, 0.2) ++(0.2, -0.2) -- ++(0, 0.2);
          \node [above] at (1.5, 3) {A};
          \node at (1.5, 1.5) {B};
          \node [below] at (1.5, 0) {C};
        \end{scope}
        &
        \begin{scope}
          \draw[loosely dotted] (4, 1) -- (3, 0.25);
          \draw (3, 0.25) -- (2, 0) -- (2, 3) node [midway, above=10pt, left=-3pt] {d} -- (3, 2.75);
          \draw (1, 0) -- (0, 0.75) -- (-0.25, 1.5) -- (0, 2.25) -- (1, 3) -- (1, 0) node [midway, above=10pt, right=-3pt] {d'};
          \draw[loosely dotted] (3, 2.75) -- (4, 2);
          \draw (2, 0) -- ++(-0.2, -0.2)  ++(0.2, 0.2) -- ++(0.2, -0.2) ++(-0.2, 0.2) -- ++(0, -0.2);
          \draw (1, 0) -- ++(-0.2, -0.2)  ++(0.2, 0.2) -- ++(0.2, -0.2) ++(-0.2, 0.2) -- ++(0, -0.2);
          \draw (0, 2.25) -- ++(-0.2, -0.2)  ++(0.2, 0.2) -- ++(-0.2, 0.2) ++(0.2, -0.2) -- ++(-0.2, 0);
          \draw (-0.25, 1.5) -- ++(-0.2, -0.2)  ++(0.2, 0.2) -- ++(-0.2, 0.2) ++(0.2, -0.2) -- ++(-0.2, 0);
          \draw (0, 0.75) -- ++(-0.2, -0.2)  ++(0.2, 0.2) -- ++(-0.2, 0.2) ++(0.2, -0.2) -- ++(-0.2, 0);
          \draw (1, 3) -- ++(0.2, 0.2)  ++(-0.2, -0.2) -- ++(-0.2, 0.2) ++(0.2, -0.2) -- ++(0, 0.2);
          \draw (2, 3) -- ++(0.2, 0.2)  ++(-0.2, -0.2) -- ++(-0.2, 0.2) ++(0.2, -0.2) -- ++(0, 0.2);
          \node at (1.5, 1.5) {D};
          \node at (3, 1.5) {E};
        \end{scope}
        \\
      };
    \end{tikzfigure}
    
    \begin{tikzfigure}{\label{fig:valence5:img5}}{}
      \matrix (m) [ column sep=1cm] {
        \begin{scope}
          \draw[loosely dotted] (4, 1) -- (3, 0.25);
          \draw (3, 0.25) -- (2, 0) -- (1, 0) -- (0, 1.5) -- (1, 3) -- (2, 3) -- (3, 2.75);
          \draw[loosely dotted] (3, 2.75) -- (4, 2);
          \draw (2, 0) -- ++(-0.2, -0.2)  ++(0.2, 0.2) -- ++(0.2, -0.2) ++(-0.2, 0.2) -- ++(0, -0.2);
          \draw (1, 0) -- ++(-0.2, -0.2)  ++(0.2, 0.2) -- ++(0.2, -0.2) ++(-0.2, 0.2) -- ++(0, -0.2);
          \draw (0, 1.5) -- ++(-0.2, -0.2)  ++(0.2, 0.2) -- ++(-0.2, 0.2) ++(0.2, -0.2) -- ++(-0.2, 0);
          \draw (1, 3) -- ++(0.2, 0.2)  ++(-0.2, -0.2) -- ++(-0.2, 0.2) ++(0.2, -0.2) -- ++(0, 0.2);
          \draw (2, 3) -- ++(0.2, 0.2)  ++(-0.2, -0.2) -- ++(-0.2, 0.2) ++(0.2, -0.2) -- ++(0, 0.2);
          \node [above] at (1.5, 3) {A};
          \node at (1.5, 1.5) {B};
          \node [below] at (1.5, 0) {C};
        \end{scope}
        &
        \begin{scope}
          \draw[loosely dotted] (4, 1) -- (3, 0.25);
          \draw (3, 0.25) -- (2, 0) -- (2, 3) node [midway, above=10pt, left=-3pt] {d} -- (3, 2.75);
          \draw (1, 0) -- (0, 1.5) -- (1, 3) -- (1, 0) node [midway, above=10pt, right=-3pt] {d'};
          \draw[loosely dotted] (3, 2.75) -- (4, 2);
          \draw (2, 0) -- ++(-0.2, -0.2)  ++(0.2, 0.2) -- ++(0.2, -0.2) ++(-0.2, 0.2) -- ++(0, -0.2);
          \draw (1, 0) -- ++(-0.2, -0.2)  ++(0.2, 0.2) -- ++(0.2, -0.2) ++(-0.2, 0.2) -- ++(0, -0.2);
          \draw (0, 1.5) -- ++(-0.2, -0.2)  ++(0.2, 0.2) -- ++(-0.2, 0.2) ++(0.2, -0.2) -- ++(-0.2, 0);
          \draw (1, 3) -- ++(0.2, 0.2)  ++(-0.2, -0.2) -- ++(-0.2, 0.2) ++(0.2, -0.2) -- ++(0, 0.2);
          \draw (2, 3) -- ++(0.2, 0.2)  ++(-0.2, -0.2) -- ++(-0.2, 0.2) ++(0.2, -0.2) -- ++(0, 0.2);
          \node at (1.5, 1.5) {D};
          \node at (3, 1.5) {E};
        \end{scope}
        \\
      };
    \end{tikzfigure}
    The step of ``cutting out'' triangles can be applied similarly as before, it will be used for the up to twelve or fifteens faces which share at least one vertex with the exceptional face (twelve if it is a square, fifteen if it is a pentagon). After these steps, assume the exceptional face to be surrounded by triangles, else one could try the above step to ``cut out'' a triangle, resulting in fewer edges or in a triangle at the desired position. By iterating this step on the faces adjacent to these twelve or fifteen faces one has a graph which is the same as the skeleton of a ``generalized icosahedron'' (these are built by a similar construction as the regular one, but with a $k$-gon as top and base face, see \autoref{fig:valence5:img6} for the resulting graphs with a quadrangle base and a pentagonal base), which has one additional exceptional face. The steps taken did not introduce new exceptional faces, a contradiction.
    \begin{tikzfigure}{\label{fig:valence5:img6}}{}
      \matrix (m) [ column sep=1cm] {
        \begin{scope}[scale=0.5]
          \draw (-4, -4) -- (4, -4) -- (4, 4) -- (-4, 4) -- cycle;
          \draw (-4, -4) -- (-3, 0) -- (-4, 4) -- (0, 3) -- (4, 4) -- (3, 0) -- (4, -4) -- (0, -3) -- cycle;
          \draw (-2, -2) -- (-3, 0) -- (-2, 2) -- (0, 3) -- (2, 2) -- (3, 0) -- (2, -2) -- (0, -3) -- cycle;
          \draw (-4, -4) -- (-2, -2) -- (0, -1) -- (2, -2) -- (4, -4);
          \draw (-4,  4) -- (-2,  2) -- (0,  1) -- (2,  2) -- (4,  4);
          \draw (-3, 0) -- (-1, 0) -- (0, -1) -- (0, -3);
          \draw ( 3, 0) -- ( 1, 0) -- (0,  1) -- (0,  3);
          \draw (-2, -2) -- (-1, 0) -- (-2, 2);
          \draw ( 2,  2) -- ( 1, 0) -- ( 2,-2);
          \draw (-1, 0) -- (0, 1) (0, -1) --(1, 0);
        \end{scope}
        &
        \begin{scope}[scale=0.75]
          \draw (0 : 3) -- (72 : 3) -- (144 : 3) -- (216 : 3) -- (288 : 3) -- cycle;
          \draw (0 : 3) -- (36 : 2) -- (72 : 3) -- (108: 2) -- (144 : 3) -- (180 : 2) -- (216 : 3) -- (252 : 2) -- (288 : 3) -- (324 : 2) -- cycle;
          \draw (0 : 2) -- (36 : 2) -- (72 : 2) -- (108: 2) -- (144 : 2) -- (180 : 2) -- (216 : 2) -- (252 : 2) -- (288 : 2) -- (324 : 2) -- cycle;
          \draw (0 : 3) -- (0 : 2) (72 : 3) -- (72 : 2) (144 : 3) -- (144 : 2) (216 : 3) -- (216 : 2) (288 : 3) -- (288 : 2);
          \draw (0 : 2) -- (36 : 1) -- (72 : 2) -- (108: 1) -- (144 : 2) -- (180 : 1) -- (216 : 2) -- (252 : 1) -- (288 : 2) -- (324 : 1) -- cycle;
          \draw (36 : 1) -- (36 : 2) (108 : 1) -- (108 : 2) (180 : 1) -- (180 : 2) (252 : 1) -- (252 : 2) (324 : 1) -- (324 : 2);
          \draw (36 : 1) -- (108 : 1) -- (180 : 1) -- (252 : 1) -- (324 : 1) -- cycle;
        \end{scope}
        \\
      };
    \end{tikzfigure}
  \end{proof}
\end{theorem}



\begin{remark}
  The cases $r=4$ and $r=5$ take similar approaches as the case $r=3$ presented in \cite{ConvexPolytopes}. There, digons are allowed. One can alter the proofs here to work in this case too, but it does not simplify the proof as it did there. Since digons are unimportant in the following, they were omitted.
\end{remark}

\begin{corollary}
  Let $r \in \set{3, 4, 5}$. There exist sequences satisfying \autoref{eq:valence:3}, \autoref{eq:valence:4} or \autoref{eq:valence:5} which cannot be $(q_3, \dots, q_n)$-$r$-realized, if 
  \begin{enumerate}[label=(\roman*)]
    \item $\gcd (k : q_k \neq 0) = 4$ or $\gcd (k : q_k \neq 0) = 5$ in case of $r = 3$, 
    \item $\gcd (k : q_k \neq 0) = 3$ in case of $r = 4$ or 
    \item $\gcd (k : q_k \neq 0) = 3$ in case of $r = 5$. 
  \end{enumerate}
  \begin{proof}
The existence of a non-realizable sequence follows from \autoref{thm:nonexistence}. If $r=3$ take the sequence $[13 \times 5, 7]$ if $\gcd = 5$ and else sequence $[8 \times 4, 10]$. For $r=4$ take the sequence $[9 \times 3, 5]$ and finally for $r=5$ take $[22 \times 3, 4]$. Any $(q_3, \dots, q_n)$-$r$-realization of these sequences would produce a polyhedron which, perceived as a planar graph, would have exactly one exceptional face, contradicting the statement of \autoref{thm:nonexistence}.
  \end{proof}
\end{corollary}
