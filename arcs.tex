\section{Patches}

Let $r \in \nats$. For the rest of this section one can think of $r$ as the given valence of a polyhedron. Many constructions in this thesis will utilize the concept of replacing each face of a polyhedron with a larger patch. It can be quite challenging to see whether the resulting structures fit together as the previous faces did. This chapter introduces the necessary formalism for these kind of constructions.

\begin{definition}[Patch] A planar graph is called a $r$-patch if every of its vertices except the ones on the outer face is $r$-valent (the exterior vertices can differ from the valence $r$, but are not required to). 
\end{definition}

\begin{definition}[Boundary structure of a patch] Each patch $\mathcal{P}$ is assigned a cyclically ordered tuple $\cyctup{w_1, \dots, w_n}$ giving values $w_k$ to each vertex $v_k$ on the outer face of $G$, where $w_k = \deg(v) - 1$ and the natural, positive order on the vertices of $\mathcal{P}$ is given. This tuple is denoted as the boundary structure of $\mathcal{P}$. Each entry is called the face count of the according vertex.
\end{definition}

The geometric interpretation is rather straightforward. To each vertex on the outer face the number of adjacent faces is associated.

If one is interested in whether two patches of a construction fit together to assemble a larger patch one has to compare the valences along some path on the boundary of each part. This is captured in the following definition:
\begin{definition}
  Two tuples $(w_1, \dots, w_n)$ and $(w'_1, \dots, w'_n)$ are said to fit together if for all $i \in \{1, \dots, n \}$ the equation $w_i + w'_{n-i} = r$ holds. Also, a single tuple $w$ is called self-fitting, if $w$ fits with itself. The boundary structure of a patch $\cyctup{w_1, \dots, w_n}$ is said to fit from $i$ to $j$ to the boundary structure of a patch $\cyctup{w'_1, \dots, w'_n}$ from $i'$ to $j'$ if $i-j \equiv i' -j' (\mod n)$ and the tuples $(w_i, \dots, w_j)$ and $(w'_{i'}, \dots, w'_{j'})$ fit. Note that these tuples could ``overflow''; if $i > j$ then $(w_i, \dots, w_j)$ denotes the tuple $(w_i, \dots, w_n, w_1, \dots w_j)$.
\end{definition}

\begin{lemma}\label{thm:fitting:arcs}
  Let $\mathcal{P}, \mathcal{P}'$ be two $r$-patches, $w = \cyctup{w_1, \dots, w_n}$ and $w' = \cyctup{w'_1, \dots, w'_m}$ their respective boundary structures and $v_k$ and $v'_k$ the vertices on the outer face associated to each $w_k$, $w'_k$. If $w_i + w'_{j'} + 1 \leq r$ and $w'_{i'} + w_{j} + 1 \leq r$ and $w$ fits from $i+1$ to $j-1$ to $w'$ from $i'+1$ to $j'-1$, then there exists a $r$-component which has $\mathcal{P}$ and $\mathcal{P}'$ as subgraphs which identifies $v_{i+k} = v'_{j'-k}$, $0 \leq k \leq (j - i \mod n)$ but not any other two vertices.
  \begin{proof}
    By ``gluing'' $\mathcal{P}$ and $\mathcal{P}'$ along the paths $v_i - \dots - v_{j}$ and $v'_{i'} - \dots - v'_{j'}$, that is, identifying $v_{i+k}$ with $v'_{j'-k}$, $0 \leq k \leq (j - i \mod n)$, the resulting graph satisfies the requirements. It is planar since both $\mathcal{P}$ and $\mathcal{P}'$ are and both graphs where combined along the outer face. Let $\tilde{v}_k$ be vertex in the new graph which represents $v_{i+k}$ and $v_{j' - k}$ from the old graphs. By construction $v_k$ is now an inner vertex if $1 \leq k \leq (j - i - 1 \mod n)$ and a vertex on the outer face if $k=0$ or $k = (j - i \mod n)$. When gluing $v_{i+k}$ with $v_{j' - k}$ ($1 \leq k \leq (j - i - 1 \mod n)$), the edges $v_{i+k-1}$--$v_{i+k}$ and $v'_{j'-k+1}$--$v'_{j'-k}$ coincide, as well as the edges $v_{i+k+1}$--$v_{i+k}$ and $v'_{j'-k-1}$--$v'_{j'-k}$. Therefore the valence of $\tilde{v}_k$ is two less than the sum of valences of $v_{i+k}$ and $v'_{j'-k}$, thus
    \begin{equation*}
      \deg(\tilde{v}_k) = \deg(v_{i+k}) + \deg(v'_{j'-k}) - 2 = w_{i+k} + 1 + w'_{j'-k} + 1 - 2 = w_{i+k} + w'_{j'-k} = r.
    \end{equation*}
    The edges $v_{i}$--$v_{i+1}$ and $v'_{j}$--$v'_{j-1}$ are now glued together, it follows $\deg(\tilde{v}_0) = \deg(v_i) + \deg(v'_{j'}) - 1 = (w_i + 1) + (w'_{j'} + 1) - 1 = w_i - w'_{j'} + 1 \leq r$. Analogous $\deg(\tilde{v}_0) \leq r$ holds. This proves that the new graph is a $r$-patch.
  \end{proof}

  TODO.
  \begin{tikzfigure}{\label{fig:patch:example}}{Glueing two patches together}
    \draw (-1, -2) -- (1, -2) (0, -2) -- (0, 2)  (-1, 2) -- (1, 2);
    \draw[dotted] (-1, -2) -- (-2, -1.5);
    \draw[dotted] (1, -2) -- (2, -1.5);
    \draw[dotted] (-1, 2) -- (-2, 1.5);
    \draw[dotted] (1, 2) -- (2, 1.5);

    \node at (0, -2) [anchor=north] {$v_i=v_{j'}$};
    \fill [black] (0, -2) circle (2pt);
    \node at (0, 2) [anchor=south] {$v_j=v_{i'}$};
    \fill [black] (0, 2) circle (2pt);
    \node at (0, -0.5) [anchor=base east] {$v_{i+k-1}=v_{j'-k+1}$};
    \fill [black] (0, -0.5) circle (2pt);
    \node at (0, 0) [anchor=base west] {$v_{i+k}=v_{j'-k}$};
    \fill [black] (0, 0) circle (2pt);
    \node at (0, 0.5) [anchor=base east] {$v_{i+k+1}=v_{j'-k-1}$};
    \fill [black] (0, 0.5) circle (2pt);
  \end{tikzfigure}
\end{lemma}

\begin{definition}
  Let $w = (w_1, \dots, w_n)$ be a tuple. The $w$-expansion of an cyclic ordered tuple $a = \cyctup{a_1, \dots a_k}$ is the cyclic ordered tuple $\cyctup{a_1, w_1, \dots w_n, a_2, w_1, \dots, w_n, a_3 \dots, a_k, w_1, \dots, w_n}$. If the boundary structure of a $r$-patch is a $w$-expansion of $a$, the vertices associated to $a_k$ will be called ``corner vertices'' while the vertices associated to any $w_k$ are called ``side vertices''.
\end{definition}

With this, one is able to describe a general construction scheme:

\begin{construction}\label{thm:construction:patch}
TODO.
% Let $G$ be a planar graph of valence $r$. Let $w = (w_1, \dots, w_n)$ be a self-fitting tuple. Let for each $k$-gonal face $f$ a $r$-patch $\mathcal{P}(f)$ be given, whose boundary structure is the $w$-expansion of the boundary structure of a $k$-gon. Subdivide each edge of $G$ by $n$ vertices - a $k$-gonal face of $G$ is now  $(n+1)k$-gonal and - replace the subdivision of $f$ with $\mathcal{P}(f)$ such that the newly inserted vertices are side vertices of $\mathcal{P}(f)$ and the original vertices are corner vertices. Each side vertex is on the boundary of two patches and has by \label{thm:fitting:arcs}
\end{construction}

\begin{remark}
  \autoref{thm:construction:patch} still holds on an orientable closed $2$-manifold, as the arguments used were only ``local planarity'' and the existence of an orientation, thus \autoref{thm:fitting:arcs} can be used for all pairs of adjacent faces. If the piece of an arc of length $n$ is also symmetric, that is, $w_i = w_{n-i}$, $i = 1, \dots n$, even the requirement of an orientation can be dropped, the necessary equations of \autoref{thm:fitting:arcs} then hold regardless of the numbering of the vertices. If $w_i$ is a weight in a symmetric self-fitting piece of an arc of length $n$, then $w_i + w_{n-i} = 2 w_i = r$, therefore for these kinds of arcs to exists $r$ has to be even; reviewing the considered cases only $r = 4$ remains.
\end{remark}
