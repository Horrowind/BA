\section{Preliminaries}

\begin{definition}[Sequence]
  A sequence $a$ in this thesis is a map $\nats \setminus \{0, 1, 2\} \rightarrow \nats$ with finite support. These maps will be written as $(a(3), a(4), ..., a(k))$, where $k = \operatorname{max} \operatorname{supp} a$. If $a$ has a small but wide support these notation will be further abbreviated as $[a(k_1) \times k_1, a(k_2) \times k_2, ...]$, where $a(k) \times k$ will be simply written as $k$ if $a(k) = 1$.
\end{definition}
\begin{example}
  $(2, 0, 0, 1)$, $[2 \times 3, 1 \times 6]$ and $[2 \times 3, 6]$ all denote the same sequence. 
\end{example}
\begin{definition}[$p$-vector and $v$-vector of an polyhedron]\label{def:relizable}
  The $p$-vector of a polyhedron $P$ is the sequence $(p_3, \dots, p_m)$, where $p_k$ denotes the number of faces with exactly $k$ vertices. Similarily the $v$-vector of $P$ is the sequence $(v_1, \dots, v_n)$, where each $v_k$ measures the number of edges of $P$ with valence $k$. 
\end{definition}

\begin{definition}
Sequences $a = (a_3, \dots, a_m)$, $b = (b_3, \dots, b_m)$ are said to be realizable, if there exists a polyhedron which has $a$ is its $p$-vector and $b$ as its $v$-vector.
\end{definition}

With this setup, we want to discuss in this thesis a broad problem arising from the combinatorical properties of polyhedra:
\begin{problem} Are two arbitrarily given sequences realizable?
\end{problem}
The most basic result answering that question partially is Eulers relation.
\begin{theorem}[Eulers relation]\label{thm:eulers:relation}
  For each polyhedron with $v$ vertices, $e$ edges and $f$ faces
  \begin{align*}
    v - e + f = 2
  \end{align*}
  holds.
\end{theorem}
Given the notation as above one can immediatly draw some basic conclusions from this relation: For one the number of vertices $v$ is obviously the sum of the entries of $v$-vector, as is the same with $f$, which is the sum of the $p$-vector. Using a double counting argument (each edge is adjacent to two faces and each edge is adjacent to two vertices) the following for the number of edges holds:
\begin{align*}
  2e = \sum_{k=3}^{m} k \cdot p_k = \sum_{k=3}^{n} k \cdot f_k
\end{align*}
Putting these equations in Eulers relation \autoref{thm:eulers:relation} gives for some $t \in (0, 1)$
\begin{align*}
  \sum_{k=3}^m v_k - \left(\frac{t}{2} \sum_{k=3}^m k v_k + \frac{1-t}{2} \sum_{k=3}^n k p_k \right) + \sum_{k=3}^n p_k = 2
\end{align*}
which can be transformed into:
\begin{align*}
  t \sum_{k=3}^m \left(\frac{2}{t} - k \right) v_k + (1-t) \sum_{k=3}^n \left( \frac{2}{1-t} - k \right) p_k = 4.
\end{align*}
This equation is independant of the value of some $v_k$ or $p_{k'}$ when the value of $t$ is appropriately chosen. Of special interest in this thesis are the cases of one $v_r$ giving no information in the above equation (so $r \geq 3$ fixed and $t = \frac{2}{r}$) and having $v_{k} = 0$ for each $k \geq 3$, $k \neq r$. The resulting polyhedron has the property that each of his vertices has valence $r$. 
% TODO Explain this.
\begin{definition}[$r$-realizable]
  A sequence $a = (a_3, \dots, a_m)$ is said to be $r$-realizable, if there exists a polyhedron such that $a$ is realizable with an $v$-vector consisting only of valence $r$ (in the sense of \autoref{def:relizable}).
\end{definition}
Under these circumstances $r$ has to be in $\{3, 4, 5\}$.

For $r \in \{3, 4, 5\}$ the following equations arise:
\begin{align}
  r &= 3: \sum_{k=3}^n \left(6 - k \right) p_k = 12 \label{eq:valence:3}\\
  r &= 4: \sum_{k=3}^n \left(4 - k \right) p_k = 8  \label{eq:valence:4}\\
  r &= 5: \sum_{k=3}^n \left( \frac{10}{3} - k \right) p_k = \frac{20}{3} \label{eq:valence:5}
\end{align}
\begin{notation}
  If an $p$-vector $(p_3, \dots, p_k)$ is meant to be seen in this context, that each vertex of the considered polyhedron has valence $r$, this valence will be appended to the sequance, i.e. $(p_3, \dots, p_k)_r$.
\end{notation}
Even in with this rather specific constraints the problem stated above has no easy definite answer, as in, not every $p$-vector satisfying on of these equations can be realized as an polyhedron.
\begin{example}
  The $p$-vector $(0, 12, 1)_3$ has no realization but satisfies \autoref{eq:valence:3}. To see this one can start with one hexagon. As only pentagons are left to build the rest of the polyhedron a ring of six pentagons has to be glued to the sides of the hexagon. Another ring of six pentagons is forced to be glued to the first, leaving a hole of the shape of an hexagon in the end, but no polygons left to fill it, see \autoref{fig:preliminaries:example}.

  \begin{tikzfigure}{\label{fig:preliminaries:example}}
\begin{scope}[xscale=1.0, yscale=0.866]
          \draw (-1, 0) -- ++(0.5, -1) -- ++(1, 0) -- ++(0.5, 1) -- ++(-0.5, 1) -- ++(-1, 0) -- ++(-0.5, -1);
          \draw (-1, 0) -- (-2, 0) -- (-2.25, -1.5) -- (-1, -2) -- (-0.5, -1);
          \draw (1, 0) -- (2, 0) -- (2.25, -1.5) -- (1, -2) -- (0.5, -1);
          \draw (-0.5, 1) -- (-1, 2) -- (0, 3) -- (1, 2) -- (0.5, 1);
          \draw (-1, -2) -- (0, -3) -- (1, -2);
          \draw (-2, 0) -- (-2.25, 1.5) -- (-1, 2);
          \draw (2, 0) -- (2.25, 1.5) -- (1, 2);
          \draw (-3, -2) -- (-3, 2) -- (0, 4) -- (3, 2) -- (3, -2) -- (0, -4) -- (-3, -2);
          \draw (-3, -2) -- (-2.25, -1.5);
          \draw (3, -2) -- (2.25, -1.5);
          \draw (-3, 2) -- (-2.25, 1.5);
          \draw (3, 2) -- (2.25, 1.5);
          \draw (0, -4) -- (0, -3);
          \draw (0, 4) -- (0, 3);
        \end{scope}   
    
  \end{tikzfigure}
  
\end{example}

\begin{problem}[General Eberhard's problem]
  Let $r \in \{3, 4, 5\}$ and $(q_3, \dots)$ be a sequence which satisfies 
  \begin{align}
    r &= 3: \sum_{k=3}^n \left( 6            - k \right) q_k = 0 \label{eq:zero:curv:3}\\
    r &= 4: \sum_{k=3}^n \left( 4            - k \right) q_k = 0 \label{eq:zero:curv:4}\\
    r &= 5: \sum_{k=3}^n \left( \frac{10}{3} - k \right) q_k = 0 \label{eq:zero:curv:5}
  \end{align}
  in the respective cases. Then show that for all sequences $(p_3, \dots)$  satisfying the respective \autoref{eq:valence:3}, \autoref{eq:valence:4} or \autoref{eq:valence:5}, there exists an $c \in \nats$ for which $(p_3 + c \cdot q_3, p_4 + c \cdot q_4, \dots)$ is realizable as a polyhedron.
\end{problem}
\begin{notation}
  Throughout the thesis we want to denote this problem as of $(p_3, \dots, p_n)_r$-type.
\end{notation}
\begin{definition}[$r$-realizable]
  A sequence $a = (p_3, \dots, p_m)$ is said to be $(q_3, \dots, q_n)_r$-realizable, if there exists a polyhedron such that $a$ is realizable with an $v$-vector consisting only of valence $r$ (in the sense of \autoref{def:relizable}).
\end{definition}

\autoref{eq:valence:3} and \autoref{eq:valence:4} suggest that $(6)_3$ or $(4)_4$ might be good candidates for such statements. As it is, these questions are well established and known as the (non-general) Eberhards theorem in \cite{ConvexPolytopes}:
\renewcommand{\Itemautorefname}{Theorem \ref{thm:eberhard}}
\begin{theorem}[Eberhards theorem] \label{thm:eberhard} The following holds:
  \begin{enumerate}[label=(\roman*)]
  \item \label{thm:eberhard:3} For each $p_3, p_4, p_5, p_7, \dots, p_m$ satisfiying \autoref{eq:valence:3} there exists a number $p_6$ such that $(p_3, \dots, p_m)$ is $3$-realizable. 
  \item \label{thm:eberhard:4} For each $p_3, p_5, \dots, p_m$ satisfiying \autoref{eq:valence:4} there exists a number $p_4$ such that $(p_3, \dots, p_m)$ is $4$-realizable. 
  \end{enumerate}
\end{theorem}

A stronger theorem, which generalizes the first case, is given in \cite{jendrol1977generalization}.

Building upon these two, this thesis will present a bunch of new theorems of general Eberhards type. One of these proofs takes a similar approach as the proof given in \cite{ConvexPolytopes} for \autoref{thm:eberhard:4}. The rest of the proofs directly use \autoref{thm:eberhard}. This is done by taking a sequence not too different form the given one which suffices either \autoref{eq:valence:3} or \autoref{eq:valence:4}, constructing a polyhedron by the statements of \autoref{thm:eberhard} and then by adding or deleting some polygons create a polyhedron with a $p$-vector as stated. This is done by either substituting edges and vertices or every face by some larger structure. 

These construction are transformation of graphs and therefore use heavily an result of Steinitz, stating that the necessary condition for an graph to be realizable as an polytope is to be planar and $3$-connected.

\begin{theorem}[Steinitz's Theorem]\label{thm:steinitz}
  A Graph $G$ is $3$-realizable if and only if $G$ is planar and $3$-connected.
\end{theorem}
A proof of this theorem can also be found in \cite{ConvexPolytopes}.
