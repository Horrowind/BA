\section{Preliminaries}

\begin{definition}[Sequence]
  A sequence $a$ in this thesis is a map $\nats \setminus \{0, 1, 2\} \rightarrow \nats$ with finite support. These maps will be written as $(a(3), a(4), ..., a(k))$, where $k = \operatorname{max} \operatorname{supp} a$. If $a$ has a small but wide support these notation will be further abbreviated as $[a(k_1) \times k_1, a(k_2) \times k_2, ...]$, where only entries not equal to zero occur. Further $a(k) \times k$ will be simply written as $k$ if $a(k) = 1$.
\end{definition}
\begin{example}
  $(2, 0, 0, 1)$, $[2 \times 3, 1 \times 6]$ and $[2 \times 3, 6]$ all denote the same sequence.
\end{example}
\begin{definition}[$p$-vector and $v$-vector of a polyhedron]\label{def:relizable}
  The $p$-vector of a polyhedron $P$ is the sequence $(p_3, \dots, p_m)$, where $p_k$ denotes the number of faces with exactly $k$ vertices. Similarly the $v$-vector of $P$ is the sequence $(v_1, \dots, v_n)$, where each $v_k$ measures the number of edges of $P$ with valence $k$.
\end{definition}

\begin{definition}\label{def:realizable}
  Sequences $a = (a_3, \dots, a_m)$, $b = (b_3, \dots, b_m)$ are said to be realizable, if there exists a polyhedron which has $a$ is its $p$-vector and $b$ as its $v$-vector.
\end{definition}

With this setup, we want to discuss in this thesis a broad problem arising from the combinatorical properties of polyhedra:
\begin{problem} Are two arbitrarily given sequences realizable?
\end{problem}
The most fundamental result giving a necessary condition is Euler's relation.
\begin{theorem}[Euler's relation]\label{thm:eulers:relation}
  For each polyhedron with $v$ vertices, $e$ edges and $f$ faces holds
  \begin{align*}
    v - e + f = 2.
  \end{align*}
\end{theorem}
Given the notation as above one can immediately draw some basic conclusions from this relation: For one the number of vertices $v$ is obviously the sum of the entries of $v$-vector, as is the same with $f$, which is the sum of the $p$-vector. Using a double counting argument (each edge is adjacent to two faces and each edge is adjacent to two vertices) the following for the number of edges holds:
\begin{align*}
  2e = \sum_{k=3}^{m} k \cdot p_k = \sum_{k=3}^{n} k \cdot f_k
\end{align*}
Putting these equations in Euler's relation \autoref{thm:eulers:relation} gives for some $t \in (0, 1)$
\begin{align*}
  \sum_{k=3}^m v_k - \left(\frac{t}{2} \sum_{k=3}^m k v_k + \frac{1-t}{2} \sum_{k=3}^n k p_k \right) + \sum_{k=3}^n p_k = 2
\end{align*}
which can be transformed into:
\begin{align}
  t \sum_{k=3}^m \left(\frac{2}{t} - k \right) v_k + (1-t) \sum_{k=3}^n \left( \frac{2}{1-t} - k \right) p_k = 4. \label{eq:general:vp:relation}
\end{align}
This equation is independent of the value of some $v_k$ or $p_{k'}$ when the value of $t$ is appropriately chosen. Of special interest in this thesis are the cases of one $v_r$ giving no information in the above equation (so $r \geq 3$ fixed and $t = \frac{2}{r}$) and having $v_{k} = 0$ for each $k \geq 3$, $k \neq r$. The resulting polyhedron has the property that each of his vertices has valence $r$.
% TODO Explain this.
\begin{definition}[$r$-realizable]\label{def:r:realizable}
  A sequence $p = (p_3, \dots, p_m)$ is said to be $r$-realizable, if $p$ is realizable by a polyhedron which has only $r$-valent vertices.
\end{definition}
Under these circumstances $r$ has to be in $\{3, 4, 5\}$, thus the following equations arise:
\begin{align}
  r &= 3: \sum_{k=3}^n \left(6 - k \right) p_k = 12 \label{eq:valence:3}\\
  r &= 4: \sum_{k=3}^n \left(4 - k \right) p_k = 8  \label{eq:valence:4}\\
  r &= 5: \sum_{k=3}^n \left( \frac{10}{3} - k \right) p_k = \frac{20}{3} \label{eq:valence:5}
\end{align}
Even in with this rather specific constraints the problem stated above has no easy definite answer, as in, not every $p$-vector satisfying on of these equations can be realized as a polyhedron.
\begin{example}
  The $p$-vector $(0, 12, 1)$ has no $3$-realization but satisfies \autoref{eq:valence:3}. To see this one can start with one hexagon. As only pentagons are left to build the rest of the polyhedron a ring of six pentagons has to be glued to the sides of the hexagon. Another ring of six pentagons is forced to be glued to the first, leaving a hole of the shape of a hexagon in the end, but no polygons left to fill it, see \autoref{fig:preliminaries:example}.

  \begin{tikzfigure}{\label{fig:preliminaries:example}}{Example for a non-realizable sequence}
    \begin{scope}[xscale=1.0, yscale=0.866]
      \draw (-1, 0) -- ++(0.5, -1) -- ++(1, 0) -- ++(0.5, 1) -- ++(-0.5, 1) -- ++(-1, 0) -- ++(-0.5, -1);
      \draw (-1, 0) -- (-2, 0) -- (-2.25, -1.5) -- (-1, -2) -- (-0.5, -1);
      \draw (1, 0) -- (2, 0) -- (2.25, -1.5) -- (1, -2) -- (0.5, -1);
      \draw (-0.5, 1) -- (-1, 2) -- (0, 3) -- (1, 2) -- (0.5, 1);
      \draw (-1, -2) -- (0, -3) -- (1, -2);
      \draw (-2, 0) -- (-2.25, 1.5) -- (-1, 2);
      \draw (2, 0) -- (2.25, 1.5) -- (1, 2);
      \draw (-3, -2) -- (-3, 2) -- (0, 4) -- (3, 2) -- (3, -2) -- (0, -4) -- (-3, -2);
      \draw (-3, -2) -- (-2.25, -1.5);
      \draw (3, -2) -- (2.25, -1.5);
      \draw (-3, 2) -- (-2.25, 1.5);
      \draw (3, 2) -- (2.25, 1.5);
      \draw (0, -4) -- (0, -3);
      \draw (0, 4) -- (0, 3);
    \end{scope}

  \end{tikzfigure}

\end{example}

\begin{definition}[$(q_3, \dots, q_n)$-$r$-realizable]\label{def:eberhard:realizable}
  A sequence $p = (p_3, \dots, p_m)$ is said to be $(q_3, \dots, q_n)$-$r$-realizable, if there exists $c \in \nats$, such that $p + c q$ is $r$-realizable.
\end{definition}
\begin{problem}[General Eberhard's problem]\label{problem:eberhard}
  Let $r \in \{3, 4, 5\}$ and $(q_3, \dots, q_n)$ be a sequence which satisfies
  \begin{align}
    r &= 3: \sum_{k=3}^n \left( 6            - k \right) q_k = 0 \label{eq:zero:curv:3}\\
    r &= 4: \sum_{k=3}^n \left( 4            - k \right) q_k = 0 \label{eq:zero:curv:4}\\
    r &= 5: \sum_{k=3}^n \left( \frac{10}{3} - k \right) q_k = 0 \label{eq:zero:curv:5}
  \end{align}
  respectively. Then show that all sequences $(p_3, \dots, p_m)$ satisfying the equations \ref{eq:valence:3}, \ref{eq:valence:4} or \ref{eq:valence:5} are $(q_3, \dots, q_n)$-$r$-realizable.
\end{problem}
\begin{notation}
  Throughout the thesis we want to denote this problem as of $(p_3, \dots, p_n)$-type for valence $r$.
\end{notation}
\autoref{eq:valence:3} and \autoref{eq:valence:4} suggest that $(6)$ might be a good candidate for a $3$-valent statement, as well as $(4)$ for $4$-valence. Indeed, in these cases the important and well-known classical Eberhard theorem in \cite{ConvexPolytopes} gives a complete answer:
\renewcommand{\Itemautorefname}{Theorem \ref{thm:eberhard}}
\begin{theorem}[Eberhard'S theorem] \label{thm:eberhard} The following holds:
  \begin{enumerate}[label=(\roman*)]
  \item \label{thm:eberhard:3} For each $p_3, p_4, p_5, p_7, \dots, p_m$ satisfying \autoref{eq:valence:3} there exists a number $p_6$ such that $(p_3, \dots, p_m)$ is $3$-realizable.
  \item \label{thm:eberhard:4} For each $p_3, p_5, \dots, p_m$ satisfying \autoref{eq:valence:4} there exists a number $p_4$ such that $(p_3, \dots, p_m)$ is $4$-realizable.
  \end{enumerate}
\end{theorem}

\begin{remark}
  There is a subtle difference in the statements of the general and the classical Eberhard theorems. The given $p$-vector excludes a minimal number of hexagons or quadrangles which is required to be in the realization, while \autoref{problem:eberhard} takes these into account. In the end, this differentiation is not important as both versions of the theorem hold. Take a realization given by the classical Eberhard theorem with no information about the minimum number of hexagons or quadrangles. To satisfy the requirements of the general problem one has to add arbitrarily hexagons in the case of valence $3$ and quadrangles in the case of valence $4$. This can be done by a repeated substitution of each edge by a hexagon or each vertex by a quadrangle as seen in \autoref{fig:edge:replacement}. In each replacement the number of hexagons or quadrangle increases, resulting in an unbounded number of hexagons or quadrangles.
  \begin{tikzfigure}{\label{fig:edge:replacement}}{
      Replacing each edge of a polyhedron by a hexagon or each vertex by a quadrangle.
      Thin-lined is the original graph (the tetrahedron left, an octahedron on the right), the new graph is drawn thick.
    }
    \matrix (m) [ column sep=1cm] {
      \begin{scope}[xscale=1.0, yscale=0.866]
        \draw (-1.5, -1) -- (1.5, -1) -- (0, 2) -- (-1.5, -1) -- (0, 0) --(1.5, -1) (0, 0) -- (0, 2);
        \draw[very thick] (-2.25, -1.5) -- (2.25, -1.5) -- (0, 3) -- (-2.25, -1.5) -- (-1.5, -1) -- (-0.75, -0.75) -- (0.75, -0.75) -- (0, -0.25) -- (-0.75, -0.75);
        \draw[very thick] (0.75, -0.75) -- (1.5, -1) -- (0.9375, -0.375) -- (0.1875, 1.125) -- (0.1875, 0.125) -- (0.9375, -0.375);
        \draw[very thick] (0.1875, 1.125) -- (0, 2) -- (-0.1875, 1.125) -- (-0.9375, -0.375) -- (-0.1875, 0.125) -- (-0.1875, 1.125)  (-0.9375, -0.375) -- (-1.5, -1);
        \draw[very thick] (1.5, -1) -- (2.25, -1.5) (0, 2) -- (0, 3);
        \draw[very thick] (0, 0) -- (0, -0.25)  (0, 0) -- (0.1875, 0.125) (-0.1875, 0.125) -- (0, 0);
      \end{scope}
      &
      \begin{scope}[xscale=1.0, yscale=0.866]
        \draw (-1.5, -1) -- (1.5, -1) -- (0, 2) -- (-1.5, -1) -- (0, -0.5) -- (1.5, -1) -- (0.375, 0.25) -- (0, 2) -- (-0.375, 0.25) -- (0.375, 0.25) -- (0, -0.5) -- (-0.375, 0.25) -- cycle;
        \draw[very thick] (0, -1) -- (-0.75, -0.75) -- (-0.9375, -0.375) -- (-0.75, 0.5) -- (-0.1875, 1.125) -- (0.1875, 1.125) -- (0.75, 0.5) -- (0.9375, -0.375) -- (0.75, -0.75) -- (0.1875, -0.125) -- (0.9375, -0.375) -- (0.1875, 1.125) -- (0, 0.25) -- (-0.1875, 1.125) -- (-0.9375, -0.375) -- (-0.1875, -0.125) -- (0, 0.25) -- (0.1875, -0.125) -- (-0.1875, -0.125) -- (-0.75, -0.75) -- (0.75, -0.75) -- (0, -1) arc[x radius = 1.5, y radius = 1.732, start angle= -180, end angle= 120] arc[x radius = 1.5, y radius = 1.732, start angle= -60, end angle=240] arc[x radius = 1.5, y radius = 1.732, start angle= 60, end angle = 360];
      \end{scope};
      \\
    };

  \end{tikzfigure}
\end{remark}

A stronger theorem, which generalizes the first case, is given in \cite{jendrol1977generalization}:

\begin{theorem}[Jendrol', Jucovi{\v{c}} 1977] \label{thm:eberhard:extended}
  Each sequence $p$ together with $v$ is realizable on the closed orientable $2$-manifold of genus $g$ for some $p_6 \in \nats$, $v_3 \in \nats$ if and only if
  \begin{align*}
    \sum_{k=3}^m (6-k)p_k + 2 \sum_{k=4}^n (3-k)v_k &= 12(1-g) \\
    \sum_{k=3,\, 2 \nmid k}^{m} p_k \neq 0 \text{ or} \sum_{k=4, \,3 \nmid k}^m v_k \neq 1 &\text{ if } g = 0 \\
    p \neq [5, 7] \text{ or } v \neq [k\times3] \text{ for all } k \in \nats &\text{ if } g = 1
  \end{align*}
\end{theorem}

Building upon these two, this thesis will present a bunch of new theorems of general Eberhard's type. One of these proofs takes a similar approach as the proof given in \cite{ConvexPolytopes} for \autoref{thm:eberhard:4}. The rest of the proofs directly use \autoref{thm:eberhard}. This is done by taking a sequence not too different form the given one which suffices either \autoref{eq:valence:3} or \autoref{eq:valence:4}, constructing a polyhedron by the statements of \autoref{thm:eberhard} and then by adding or deleting some polygons create a polyhedron with a $p$-vector as stated. This is done by either substituting edges and vertices or every face by some larger structure.

These construction are transformation of graphs and therefore use heavily a result of Steinitz, stating that the necessary condition for a graph to be realizable as a polytope is to be planar and $3$-connected.

\begin{theorem}[Steinitz's Theorem]\label{thm:steinitz}
  A Graph $G$ is realizable as a polyhedron if and only if $G$ is planar and $3$-connected.
\end{theorem}
A proof of this theorem can also be found in \cite{ConvexPolytopes}.
